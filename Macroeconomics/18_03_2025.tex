\documentclass[12pt,a4paper]{article}
\usepackage{import}
\usepackage{pdfpages}
\usepackage{transparent}
\usepackage{xcolor}
\usepackage{setspace}
\usepackage[a4paper, left=10mm, right=10mm]{geometry}
\setstretch{1.1}
\newcommand{\incfig}[2][1]{%
\def\svgwidth{#1\columnwidth}
\import{./figures/}{#2.pdf_tex}}
\pdfsuppresswarningpagegroup=1
\usepackage[utf8]{inputenc}
\usepackage[T1]{fontenc}
\usepackage{parskip}
\usepackage{amsmath, amssymb, graphicx}
\usepackage{tcolorbox}
\usepackage{fancyhdr}
\setlength{\headheight}{15.6pt}
\pagestyle{fancyplain}
\fancyhf{}
\renewcommand{\footrulewidth}{1pt}
\lfoot{Laxman Singh}
\cfoot{\thepage}
\rfoot{\today}
\usepackage{float}
\floatstyle{boxed}
\restylefloat{figure}
\author{Laxman Singh}
\date{\today}


\lhead{Macroeconomics}
\rhead{SGM, AD-AS}
\graphicspath{{/Users/econhead/Econschool-notes/Macroeconomics/}}
\begin{document}
\section{\underline{Aggregate Demand and Aggregate Supply Model}}
The next four questions are based on the following information. Consider an economy with an aggregate production function \(Y=\alpha K+\beta L\), where \(\alpha\) and \(\beta\) are positive constants, \(K\) is capital, \(L\) is labour and \(Y\) is output. \(K\) is fixed in the short run. Perfectly competitive producers take the nominal wage rate \(W\) and the price level \(P\) as given, and employ labour so as to maximize profit. This generates the labour demand schedule. The labour supply schedule is \(L^S=-\gamma+\delta W / P\), where \(\gamma\) and \(\delta\) are positive constants. Producers and workers have perfect information about \(P\) and \(W\).

\subsection{Question}
The labour  market will clear at a positive level of employment?
The firm will first solve the short run profit maximization problem and derive the labour demand curve,
\begin{align*}
    \max_{L }  & \quad p(\alpha K + \beta L)
  \end{align*}
\[L^d \in \begin{cases}
  \phi & \text{if} \quad p\beta > W\\
  \mathbb{R}_{+} & \text{if} \quad p\beta = W\\
  \{0\} & \text{if} \quad p\beta < W\\
\end{cases}\] 
\[L^s = - \gamma + \frac{\delta W}{P}\] 
\begin{figure}[H]
\centering
\includegraphics[scale=0.9]{1.png}
\end{figure}
\subsection{Question}
Assume that the required parametric condition from the previous question holds and that the nominal wage is rigid and it is fixed. The short run aggregate supplyfor this economy, with P along the vertical axis and \(Y\) along the horizontal axis will look as follows:

(a) for high values of \(P\) it will be horizontal; for some mid-range values of \(P\) it will be downward sloping; for low values of \(P\) it will be horizontal again

(b) for high values of \(P\) it will be horizontal; for some mid-range values of \(P\) it will be upward sloping; for low values of \(P\) it will be horizontal again

(c) for high values of \(P\) it will be vertical; for some mid-range values of \(P\) it will be downward sloping; for low values of \(P\) it will be vertical again

(d) for high values of \(P\) it will be vertical; for some mid-range values of \(P\) it will be upward sloping; for low values of \(P\) it will be vertical again

\begin{figure}[H]
\centering
\includegraphics[scale=0.5]{2.png}
\end{figure}

\subsection{Question}
If there is a one shot increase in the fixed stock of the capital stock, then the short run aggregate supply schedule will

(a) shift up

(b) shift down

(c) shift to the left

(d) shift to the right

The short run aggregate supply curve will shift to the right.

\subsection{Question}
If there is a one shot increase in the fixed nominal wage rate, then the short run aggregate supply schedule will

(a) shift up

(b) shift down

(c) shift to the left

(d) shift to the right

The short run aggregate supply curve will shift up.

\subsection{Question}

Consider an aggregate demand and aggregate supply model where, in the short run, aggregate capital is fixed at the level \(\bar{K}\). The aggregate demand curve, aggregate output \((Y)\) demanded as a function of aggregate price level \((P)\), is given by a standard downward-sloping curve. The aggregate supply curve, aggregate output \((Y)\) supplied as a function of aggregate price level \((P)\), is not standard, and the question leads you to derive the aggregate supply curve.

The aggregate production function is linear in capital and labour \((L): Y = \min(A L , \overline{K}), A>0\). Assume that \(\overline{K}< A \overline{K}\). The labour union is very powerful and dictates the minimum aggregate nominal wage rate as \(\bar{W}\). Each worker is endowed with one unit of labour which they supply inelastically if the producers offer the nominal wage \(W>\bar{W}\). A worker does not supply any labour if \(W<\bar{W}\). At \(W=\bar{W}\), a worker is indifferent between supplying and not supplying her labour endowment. The number of workers available in the economy is fixed at \(\bar{L}\).

(a) Derive, with a clear explanation, the aggregate labour supply \(\left(L^S\right)\) in this economy as a function of the aggregate nominal wage rate, \(W\).

(b) Note that the marginal product of labour is constant, \(A>0\).

(i) Derive, with a clear explanation, the aggregate labour demand \(\left(L^D\right)\) in this economy as a function of the real wage rate, \(\frac{W}{P}\).

(ii) Using your answer to part (i) above, derive the aggregate labour demand \(\left(L^D\right)\) in this economy as a function of the aggregate nominal wage rate, \(W\).

(c) Choose an arbitrary aggregate price level, \(P\), and draw the aggregate labour supply \(\left(L^S\right)\) and aggregate labour demand \(\left(L^D\right)\) curves, as functions of \(W\), by plotting labour \((L)\) on \(x\)-axis and nominal wage \((W)\) on \(y\)-axis. Think about the labour market equilibrium for the arbitrary aggregate price level \(P\) that you have chosen.

Note that the equilibrium employment ( \(L^*\) ) in the economy depends on the arbitrary price level \(P\) that you choose. Derive, with a clear explanation, the equilibrium employment \(\left(L^*\right)\) as a function of aggregate price level \(P\).

(d) Derive, with a clear explanation, aggregate output \((Y)\) supplied as a function of aggregate price level \((P)\). Draw this aggregate supply curve by plotting \(Y\) on \(x\)-axis and \(P\) on \(y\)-axis.

(e) Recall that the aggregate demand curve is given by a standard downward-sloping curve. Explain the effectiveness of the standard monetary and fiscal policies in this set up.

 \subsection*{Solution}

\(L^s \in \begin{cases}
  \{\overline{L}\} & \text{if} \quad W > \overline{W}\\
  [0 , \overline{L}] & \text{if} \quad W = \overline{W}\\
  \{0\} & \text{if} \quad W < \overline{W}
\end{cases}\) 

 The firm then solves the profit maximization problem to find the labour demand schedule,
 \begin{align*}
     \max_{L \geq 0 }  & \quad p \min(AL , \overline{K}) - wL - F
 \end{align*}
 \[ L^d \in \begin{cases}
   \{ \frac{\overline{K}}{A}\} & \text{if} \quad W<pA\\
   [0 , \frac{\overline{K}}{A}] & \text{if} \quad W=pA\\
   \{0\} & \text{if} \quad W>pA
 \end{cases} \] 
 The equilibrium level of employment will be;

 \begin{figure}[H]
 \centering
 \includegraphics[scale=0.9]{3.png}
 \end{figure}


 \begin{figure}[H]
 \centering
 \includegraphics[scale=0.9]{4.png}
 \end{figure}

 \begin{figure}[H]
 \centering
 \includegraphics[scale=0.9]{5.png}
 \end{figure}
Mathematically the equilibrium level of employment in the economy is;
\[L^* \in \begin{cases}
  \{\frac{\overline{K}}{A}\}  & \text{if} \quad p > \frac{\overline{W}}{A} \\
  [0 , \frac{\overline{K}}{A}] & \text{if} \quad p = \frac{\overline{W}}{A} \\
  \{0\}  & \text{if} \quad p < \frac{\overline{W}}{A} \\
\end{cases}\] 
The Aggregate Supply curve is the following;
\[Y(p) \in \begin{cases}
  \{\overline{K}\}  & \text{if} \quad p > \frac{\overline{W}}{A} \\
  [0 , \overline{K}] & \text{if} \quad p = \frac{\overline{W}}{A} \\
  \{0\}  & \text{if} \quad p < \frac{\overline{W}}{A} \\
\end{cases}\] 

\begin{figure}[H]
\centering
\includegraphics[scale=0.9]{6.png}
\end{figure}

 \section{\underline{Solow Growth model} }
\subsection{Question}
Consider the following version of the Solow growth model. The aggregate output at time \(t, Y_t\), depends on the aggregate capital stock \(\left(K_t\right)\) and aggregate labour force \(\left(L_t\right)\) in the following way:

\begin{align*}
Y_t=\left(K_t\right)^\alpha\left(L_t\right)^{1-\alpha}, 0<\alpha<1
\end{align*}


There is perfect competition in the factor market so that, in equilibrium, each factor is paid its marginal product and the total output is distributed to all the households in the form of wage earnings and interest earnings. Households save a proportion \(0<s<1\) of their disposable income in every period. All household savings are invested which augment the capital stock over time. There is no depreciation of capital. Population and
5
therefore the aggregate labour force grows at a constant rate \(n>0\).

(a) The government taxes the interest earnings at the rate \(0<\tau<\) 1. Wage earnings are not taxed. The government uses the collected taxes to fund government consumption; in particular, the tax collection is not used for investment at all.

(i) Derive, with clear explanations, the expressions for aggregate wage earning, aggregate interest earning and aggregate savings \(\left(S_t\right)\) of the economy in terms of \(Y_t\).

(ii) Define \(k_t \equiv \frac{K_t}{L_t}\), the capital-labour ratio in period \(t\). Derive, with a clear explanation, the law of motion of capital-labour ratio, that is, the equation with \(k_{t+1}\) on the left-hand side and \(k_t\) on the right-hand side.

(iii) Derive, with a clear explanation, the steady-state level of capital-labour ratio in this economy, \(k^*\), and examine how \(k^*\) changes with changes in the tax rate \(\tau\).

(b) As in part (a) above, the government continues taxing interest earnings at the rate \(\tau\) and wage earnings are not taxed. But consider now that the tax revenue collected is used to fund investment by the government so that the capital stock is further augmented by this public investment.
(i) Derive, with a clear explanation, the expression for aggregate investment in this economy.
(ii) Derive, with a clear explanation, the new law of motion of capital-labour ratio.

(iii) Derive the new steady-state level of capital-labour ratio in this economy, \(k^{* *}\), and compare it with \(k^*\). Does the comparison make economic sense?

(iv) How does \(k^{* *}\) change with changes in \(\tau\) ? Compare with the response of \(k^*\) and explain the economic reason behind the differential impact.
\subsection*{Solution}
\(s_{t} = sY_{t} \), 

\(w_{t} = (1-\alpha)(\frac{K_{t} }{L_{t} })^{\alpha}\),

Wage earnings will be, \(w_{t} L_{t} - (1-\alpha) Y_{t} \),

Interest earnings will be, \(r_{t} K_{t}= \alpha Y_{t}  \) , 

\(K_{t+1} = K_{t} + s(Y_{t} - \tau \alpha Y_{t} )  \) 

\(\frac{L_{t+1} }{L_{t} } \times \frac{K_{t+1} }{L_{t+1} } = \frac{K_{t} }{L_{t} }+ s(1 - \tau \alpha)(\frac{K_{t} }{L_{t} })^\alpha\),

\((n+1)k_{t} = k_{t} + s(1-\tau \alpha)k_{t}^\alpha  \),

\(k_{t+1} = k_{t} = k^*\) 




\end{document}
