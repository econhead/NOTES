\documentclass[12pt,a4paper,fleqn]{article}

\usepackage[utf8]{inputenc}
\usepackage[T1]{fontenc}
\usepackage{parskip}
\usepackage{amsmath, amssymb, graphicx}
\usepackage[most]{tcolorbox}
\usepackage{fancyhdr}
\setlength{\headheight}{15.6pt}
\pagestyle{fancyplain}
\fancyhead[L]{Laxman Singh}
\fancyhead[R]{\today}
\usepackage{float}
\floatstyle{boxed}
\restylefloat{figure}

\author{Laxman Singh}
\date{\today}
\title{Public Goods}

\begin{document}
\section{\underline{Public Goods}}
\subsection{\underline{Dolbear's Triangle}}
 \begin{align*}
  &u_{1}(x_{1}, g)\\
  &u_{2}(x_{2},g)\\
  &\mathcal{F}=\{(x_{1},x_{2},g) \in \mathbb{R}^3_{+} \mid \left( 10-(x_{1}+x_{2}) \right) = g\}  \\
  & \ \ =\{(x_{1},x_{2},g)\in \mathbb{R}^3_{+} \mid x_{1}+x_{2}+g=10\} \\
  &\omega_{1}=\omega_{2}=5\
 \end{align*} 
\subsubsection{Question}
      \begin{align*}
         u_{1}(x_{1},g)&=8x_{1}+5g\\
         u_{2}(x_{2},g)&=8x_{2}+5g\\
         \mathcal{F}&=\{(x_{1},x_{2},g)\in \mathbb{R}^3_{+} \mid 10-(x_{1}+x_{2})=g \} \\
         u_{2}^{ADJ}(x_{1},g) &= u_{2}(10-x_{1}-g,g)\\
         &= 80-8x_{1}-3g\\
         PE &= \{(x_{1},x_{2},g) \in \mathcal{F} \mid x_{1}=0 \vee x_{2}=0\}
     \end{align*}
 Now we can solve the following problem
 \begin{tcolorbox}
 \begin{align*}
        \max_{(x_{1},x_{2},g)\in \mathbb{R}^{3}_{+}} & \alpha(8x_{1}+5g) + \beta(8x_{2}+5g) \quad   \\
        s.t. & \quad x_{1}+x_{2}+g=10
 \end{align*}
 \end{tcolorbox}
 The above problem is equivalent to solving
      \begin{align*}
            \max_{(x_{1},x_{2},g)\in \mathbb{R}^{3}_{+}} & \quad 8 \alpha x_{1} + 8 \beta x_{2} + 5(\alpha+\beta)g \quad  \\
            s.t. & \quad x_{1}+x_{2}+g=10 \quad 
     \end{align*}
 Note that
      \begin{itemize}
        \item If \(\alpha>\beta \) then \( x_{2}=0\)
        \item if \(\alpha<\beta \) then \( x_{1}=0\)
        \item if \(\alpha=\beta \) then \( x_{1}=x_{2}=0, g=10\) 
      \end{itemize}
\pagebreak
 Now
 \begin{itemize}
    \item if \( \alpha > \beta \) and \(8\alpha > 5(\alpha+\beta)\) then,\(x_{1}=10, x_{2}=0, g=0 \)
    \item if \(\alpha > \beta \) and \( 8\alpha < 5(\alpha+\beta)\) then, \(x_{1}=0, x_{2}=0, g=10 \) 
    \item if \(\alpha > \beta \) and \( 8\alpha = 5(\alpha+\beta)\) then, \( x_{2}=0, x_{1}+ g=10 \)    
 \end{itemize} 
Similiarly;
\begin{itemize}
    \item if \( \alpha < \beta \) and \(8\beta > 5(\alpha+\beta)\) then,\(x_{1}=0, x_{2}=10, g=0 \)
    \item if \(\alpha < \beta \) and \( 8\beta < 5(\alpha+\beta)\) then, \(x_{1}=0, x_{2}=0, g=10 \) 
    \item if \(\alpha < \beta \) and \( 8\beta = 5(\alpha+\beta)\) then, \( x_{1}=0, x_{2}+ g=10 \)    
 \end{itemize} 

\subsection{\underline{Lindahl equilibrium}}
We have to find the equilibrium price ratio \(\left( p_{1}^*,p_{2}^* \right) \);
\subsubsection{Question} 
Find the competitive equilibrium in the following economy;
 \begin{align*}
    u_{1}(x_{1},g)&= 8x_{1} + 5g \qquad \omega_{1}=4 \\
    u_{2}(x_{2},g)&= 8x_{2} + 5g \qquad \omega_{2}=6 \\
    g&=f(x_{0})= x_{0}\\
    \mathcal{F}&= \{(x_{1},x_{2},g) \in \mathbb{R}^3_{+} | x_{1}+x_{2}+g=10\} \\
    \left( p_{1}^*,p_{2}^* \right) = \left( \frac{2}{5},\frac{3}{5} \right) \\
\end{align*}
How to check if it is correct?

To solve for the equilibrium price ratio, we can solve the following two problems;
\begin{tcolorbox}
    \begin{align*}
        \max_{(x_{1},g)\in \mathbb{R}^{2}_{+}} & \quad 8x_{1} + 5g \\
        s.t. &  \quad x_{1} + \frac{2}{5}g=4
  \end{align*}   
\end{tcolorbox}
Solution to the above problem is \(\left( x_{1},x_{2},g \right) = \left( 0,0,10 \right) \) since \(\frac{8}{5}<\frac{5}{2}\).   

Similiarly

\begin{tcolorbox}
    \begin{align*}
        \max_{(x_{2},g)\in \mathbb{R}^{2}_{+}} & \quad 8x_{2} + 5g \\
        s.t. &  \quad x_{2} + \frac{3}{5}g=6
  \end{align*}   
\end{tcolorbox}
Solution to the above problem is \(\left( x_{1},x_{2},g \right) = \left( 0,0,10 \right) \) since \(\frac{8}{5}<\frac{5}{3}\).   

 \subsubsection{First Welfare Theorem for Lindahl Equilibrium;} 
 \begin{align*}
    &u_{1}(x_{1},g) \\
    &u_{2}(x_{2},g) \\
    &g=f(x_{0}) \\
    &(\omega_{1},\omega_{2}) \\
    &(\theta_{1},\theta_{2})\\
    &\mathcal{F}= \{(x_{1},x_{2},g) \in \mathbb{R}^3_{+} | f(\omega_{1}+ \omega_{2} -x_{1}-x_{2})=g\} \\
    &u_{1}, u_{2} \ \text{are increasing by assumption}
\end{align*}

\begin{tcolorbox}
 \underline{Proof} Suppose \(\left( (p_{1}^*,p_{2}^*),(x_{1}^*,x_{2}^*,g^*) \right)\)  is the Lindahl equilibrium, and suppose \(\left( x_{1}^*,x_{2}^*,g^* \right)\) is not Pareto efficient.
 
 So there must exist \((x_{1}',x_{2}',g') \in \mathcal{F}\) such that,

 \begin{enumerate}
    \item \(u_{1}(x_{1}',g') \geq u_{1}(x_{1}^*,g^*) \ \text{and} \ u_{2}(x_{2}',g') \geq u_{2}(x_{2}^*,g^*)\)
    \item \(u_{1}(x_{1}',g') > u_{1}(x_{1}^*,g^*) \ \text{or} \ u_{2}(x_{2}',g') > u_{2}(x_{2}^*,g^*)\)
 \end{enumerate}
 
 By 1, \(x_{1}'+p_{1}^*g' \geq \omega_{1} + \theta_{1}\pi^*(p_{1}^*+p_{2}^*)\) and \(x_{2}'+ p_{2}^*g' \geq \omega_{2} + \theta_{2}\pi^*(p_{1}^*+p_{2}^*)\) 

or

 By 2, \(x_{1}'+p_{1}^*g' > \omega_{1} + \theta_{1}\pi^*(p_{1}^*+p_{2}^*)\) and \(x_{2}' +p_{2}^*g' > \omega_{2} + \theta_{2}\pi^*(p_{1}^*+p_{2}^*)\) 

 So, adding the above two conditions which we got form 2, we get

 \(x_{1}' + x_{2}' + (p_{1}^* + p_{2}^*) g' > \omega_{1} + \omega_{2} + \pi^*(p_{1}^* + p_{2}^*)\)  

 \(\implies \pi^* (p_{1}^* + p_{2}^*) < (p_{1}^* + p_{2}^*)g' - (\omega_{1} +\omega_{2} - x_{1}' -x_{2}')\)  

 But this is a contradiction, hence it must be the case that \((x_{1}^*, x_{2}^*,g^*)\) is infact Pareto efficient.  
\end{tcolorbox}

\subsubsection{Question 2}
Find the Lindahl Equilibrium;
  \begin{align*}
     u_{1}(x_{1},g)&= x_{1}+2\sqrt{g} \qquad \omega_{1}=2\\
     u_{2}(x_{2},g)&= x_{2}+4\sqrt{g} \qquad \omega_{2}=1\\
     u_{3}(x_{3},g)&= x_{2}+4\sqrt{g} \qquad \omega_{3}=1\\
     g&=f(x_{0})=x_{0}
\end{align*}
First we solve the firm's profit maximization problem;
\begin{tcolorbox}
    \begin{align*}
        \max_{(g,x_{0})} & \quad (p_{1}+p_{2}+p_{3})g-x_{0} \\
        &s.t. \ g \leq x_{0}
 \end{align*}
\end{tcolorbox}
Which is equivalent to solving,
     \begin{equation*}
           \max_{(g)} (p_{1}+p_{2}+p_{3})g-g
    \end{equation*}
 So the solution to the above problem is,
    \begin{equation*}
        g* \in \begin{cases} \phi & \text{if} \ p_{1}+p_{2}+p_{3} > 1\\
            \{0\} & \text{if} \ p_{1}+p_{2}+p_{3} <1\\
            \mathbb{R}_{+} & \text{if} \ p_{1}+p_{2}+p_{3}=1
        \end{cases}
    \end{equation*}    

Now we can solve the utility maximization probelms of the agents;
\begin{tcolorbox}
     \begin{align*}
           \max_{(x_{1},g)\in \mathbb{R}^{2}_{+}} & \quad x_{1} + 2\sqrt{g} \\
           s.t. & \quad x_{1} + p_{1}^* g =2
    \end{align*}
\end{tcolorbox}
Which is equivalent to solving;
\begin{align*}
       \max_{0 \leq g_{1} \leq \frac{2}{p_{1}^*}} & \quad 2-p_{1}^*g_{1} + 2\sqrt{g_{1}} \\
\end{align*}
Differentiatig w.r.t \(g_{1}\)  we get \(-p_{1}^* + \frac{1}{\sqrt{g_{1}}}\)  

So the solution to the above problem is,
\begin{equation*}
    g_{1} = \begin{cases} {\left( \frac{1}{p_{1}^*} \right)}^{2} & \ \text{if} \ p_{1}^* > \frac{1}{2}\\
        \frac{2}{p_{1}^*} & \ \text{if} \ p_{1}^* \leq \frac{1}{2} 
    \end{cases}
\end{equation*}     
\begin{tcolorbox}
    \begin{align*}
          \max_{(x_{2},g)\in \mathbb{R}^{2}_{+}} & \quad x_{2} + 4\sqrt{g} \\
          s.t. & \quad x_{2} + p_{2}^* g =1
   \end{align*}
\end{tcolorbox}
Which is equivalent to solving;
\begin{align*}
       \max_{0 \leq g_{2} \leq \frac{1}{p_{2}^*}} & \quad 1-p_{2}^*g_{2} + 4\sqrt{g_{2}} \\
\end{align*}
Differentiatig w.r.t \(g_{2}\)  we get \(-p_{2}^* + \frac{2}{\sqrt{g_{2}}}\)  

So the solution to the above problem is,
\begin{equation*}
    g_{2} = \begin{cases} {\left( \frac{2}{p_{2}^*} \right)}^{2} & \ \text{if} \ p_{2}^* > 4\\
        \frac{1}{p_{2}^*} & \ \text{if} \ p_{2}^* \leq 4
    \end{cases}
\end{equation*}    
\begin{tcolorbox}
    \begin{align*}
          \max_{(x_{3},g)\in \mathbb{R}^{2}_{+}} & \quad x_{3} + 4\sqrt{g} \\
          s.t. & \quad x_{3} + p_{3}^* g =1
   \end{align*}
\end{tcolorbox}
Which is equivalent to solving;
\begin{align*}
       \max_{0 \leq g_{3} \leq \frac{1}{p_{3}^*}} & \quad 1-p_{3}^*g_{3} + 4\sqrt{g_{3}} \\
\end{align*}
Differentiatig w.r.t \(g_{3}\)  we get \(-p_{3}^* + \frac{2}{\sqrt{g_{3}}}\)  

So the solution to the above problem is,
\begin{equation*}
    g_{3} = \begin{cases} {\left( \frac{2}{p_{3}^*} \right)}^{2} & \ \text{if} \ p_{3}^* > 4\\
        \frac{1}{p_{3}^*} & \ \text{if} \ p_{3}^* \leq 4
    \end{cases}
\end{equation*} 
Now we can equate the quantity demanded and quantity supplied of the public good equal to solve for the equilibrium price ratio;

We know that for supply of the public good to be positive we need \(p_{1}^* + p_{2}^* +p_{3}^* =1 \) or simply because of the preferences and endowments of agent 2 and 3 \(p_{1}^* + 2p_{2}^*=1 \ldots (i)\).

So if \(g_{1}^*=\frac{2}{p_{1}^*}\) and \(g_{2}=g_{3}=\frac{1}{p_{2}^*}=\frac{1}{p_{3}^*}\) then equating \(g_{1}=g_{2}=g_{3}\) we get \(\frac{2}{p_{1}^*}=\frac{1}{p_{2}^*}=\frac{1}{p_{3}^*}\) or \(p_{1}^*=2p_{2}^* \ldots (ii)\)

now using (i) and (ii) we get \(p_{2}^*=\frac{1}{4}\) and therfore \( p_{1}^*=\frac{1}{2}\) and since \(p_{2}^*=p_{3}^*\) we also get \(p_{3}^*=\frac{1}{4}\).

substituing for \(p_{1}^*, p_{2}^*, p_{3}^* \) in \(g_{1},g_{2},g_{3}\) we get that \(g_{1}^*=g_{2}^*=g_{3}^*=4\)      

\end{document}