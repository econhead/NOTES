\documentclass[12pt,a4paper]{article}

\usepackage[utf8]{inputenc}
\usepackage[T1]{fontenc}
\usepackage{parskip}
\usepackage{amsmath, amssymb, graphicx}
\usepackage{tcolorbox}
\usepackage{fancyhdr}
\setlength{\headheight}{15.6pt}
\pagestyle{fancyplain}
\fancyhead[L]{Laxman Singh}
\fancyhead[R]{\today}
\usepackage{float}
\floatstyle{boxed}
\restylefloat{figure}

\author{Laxman Singh}
\date{\today}
\title{Bayesian Games Continued}

\begin{document}
 \section*{\underline{Bayesian Games (Contd.)}}
   \subsection*{  \underline{ First Price Auction } } 

 Two bidders are trying to purchase the same item in a sealed bid auction. The bidders simultaneously submit bids \(b_1\) and \(b_2\) and the auction is sold to the highest bidder at his bid price (this is called a "first price" avetion). If there is a tie, there is a coin flip to determine the winner. Suppose the players utilities are:
 
 \begin{align*}
 u_i\left(b_i, b_{-i}, v_i, v_{-i}\right)= \begin{cases}v_i-b_i & \text { if } b_j>b_{-i} \\ \frac{1}{2}\left(v_i-b_j\right) & \text { if } b_j=b_{-i} \\ 0 & \text { if } b_i<b_{-i}\end{cases}
 \end{align*}
 
 
 The key informational feature is the each player knows his own value for the item (i.e. bidder \(i\) knows \(v_i\) ), but does not know the valuation of his rival. Instead, we assume that each bidder had a prior belief that his rival's valuation \(v_j\) is a draw from a distribution \(F_j\) on \(\mathrm{R}_1\), and that these prior beliefs are common knowiedge.
 
  \subsubsection*{ Example 1(First Price Auctions, } 
  - Set of players is \(N=\{1,2\}\)
  
  - Action Sets; \(A_{1}=A_{2}=\mathbb{R}_{+}\)
  
  - Type Sets; \(\Theta_{1}=\Theta_{2}=[0,1]\)
  
  - Utility functions; \(u_{1}:A_{1} \times A_{2} \times \Theta_{1} \times \Theta_{2} \to \mathbb{R}\) 
  
   \begin{align*}
      u_{1}(b_{1},b_{2},v_{1},v_{2})=
      \begin{cases}
        v_{1}-b_{1} & \text{if} \ b_{1}>b_{2} \\
        \frac{1}{2}(v_{1}-b_{1}) & \text{if} \ b_{1}=b_{2}\\
        0 & \text{if} \ b_{1}<b_{2}\\
      \end{cases}
  \end{align*}

Similiarly for 2 

- \(v_{1}\) and \(v_{2}\) are \(i.i.d \text{Unif}(0,1)\) and 1's belief about 2's type \(v_{2}\) is that \(v_{2} \sim \text{Unif}(0,1)\) and 2's belief is that \(v_{1} \sim \text{Unif}(0,1)\).

Suppose now that player 2's strategy is \(b_{2}: \Theta_{2} \to A_{2}\), 
\begin{equation*}
    b_{2}(v_{2})=\frac{v_{2}}{2}
\end{equation*} 

What is the expected payoff of player 1? and them maximize the expected payoff w.r.t \(b_{1} \geq 0\).
\begin{align*}
    U_{1}(b_{1},b_{2}(v_{2}),v_{1})&= \left( v_{1}-b_{1} \right)\Pr(v_{2}<2b_{1}) + \frac{1}{2}(v_{1}-b_{1})\Pr(v_{2}=2b_{1}) + 0\Pr(v_{2}>2b_{1})\\
    &=\begin{cases}
        v_{1}-b_{1} & \text{ if} b_{1}>\frac{1}{2}\\
        2b_{1}\left( v_{1}-b_{1} \right) & \text{if} b_{1} \leq \frac{1}{2} 
    \end{cases}
\end{align*}  

Now we can Solve;  
\begin{align*}
    \max_{b_{1} \geq 0} \ U_{1}(b_{1},b_{2}(\cdot),v_{1})
\end{align*}
Or,
\begin{align*}
    \max_{0 \leq b_{1} \leq \frac{1}{2}} \ 2b_{1}(v_{1}-b_{1})\\
    \implies b_{1}(v_{1})= \frac{v_{1}}{2}
\end{align*}

Similiar reasoning holds for \(v_{2}(b_{2})\)  and therefore,
\begin{equation*}
    \begin{split}
        b_{1}(v_{1})=\frac{v_{1}}{2}\\
        b_{2}(v_{2})=\frac{v_{2}}{2}
    \end{split}
\end{equation*}     
is a Bayesian Nash Equilibrium for this game.

Now if we think from the seller's perspective and solve for bayesian nash equilibrium where the bidders with valuations \(v_{1},v_{2} \overset{iid}{\sim}\text{Unif}(0,1)\)   and bid according to \(b_{1}(v_{1})=\frac{v_{1}}{2}\) and \(b_{2}(v_{2})=\frac{v_{2}}{2}\), What is the expected revenue for the seller in a first price auction? 

Note that the revenue of the seller is \(\max(b_{1},b_{2})\) or 
\begin{equation*}
    \begin{split}
        \max(\frac{v_{1}}{2},\frac{v_{2}}{2})\\
        \frac{1}{2}\max(v_{1},v_{2})
    \end{split}
\end{equation*}    
 
now since \(v_{1},v_{2} \overset{iid}{\sim} \text{Unif}(0,1)\);

\(\mathbb{E}\left( \frac{\max(v_{1},v_{2})}{2} \right) = \frac{1}{3} \).  

  \subsection*{  \underline{ Second Price Auction }  } 

Two bidders are trying to purchase the same item in a sealed bid auction. The bidders simultaneously submit bids \(b_1\) and \(b_2\) and the auction is sold to the highest bidder at the other bidder's bid price (this is called a "second price" auction). If there is a tie, there is a coin flip to determine the winner. Suppole the players utilities are:

\begin{align*}
u_i\left(b_i, b_{-i}, v_i, v_{-i}\right)= \begin{cases}v_i-b_{-i} & \text { if } b_i>b_{-i} \\ \frac{1}{2}\left(v_i-b_i\right) & \text { if } b_i=b_{-i} \\ 0 & \text { if } b_i<b_{-i}\end{cases}
\end{align*}


The key informational feature is the each player knows his own value for the item (i.e. bidder \(i\) knows \(v_i\) ), but does not know the valuation of his rival. Instead, we assume that each bidder had a prior belief that his rival's valuation \(v\) is a draw from a distribution \(F_j\) on \(\mathbb{R}_{+}\), and that these prior beliefs are common knowledge.

 \subsubsection*{  \underline{ Example 2 (Second price auction(Private value)) } } 

Notice that It is a weakly dominant strategy of player to bid his own valuation.

\(U_{1}(v_{1},b_{2}(\cdot),v_{1}) \geq U_{1}(b_{1},b_{2}(\cdot), v_{1}) \ \forall b_{1} \ \forall b_{2}(\cdot) \ \forall v_{1}\).

\begin{align*}
    \begin{array}{c|c|c|c} 
    & b_{1}^{'} \ < & \ v_{1} \ < &\ b_{1}^{''}  \\
    \hline \Pr\left(b_2 \leq b_1^{\prime}\right) & v_1-b_2 & v_1-b_2 & v_1-b_2 \\
    \hline \Pr\left.r_1^{\prime}<b_2 \leq v_1^{\prime}\right) & 0 & v_1-b_2 & v_1-b_2 \\
    \hline \Pr\left(v_1<b_2<b_1^{\prime \prime}\right) & 0 & 0 & v_1-b_2 \\
    \hline \Pr\left(b_2>b_1^{\prime \prime}\right) & 0 & 0 & 0
    \end{array}
    \end{align*}

 \section*{  \underline{ Market for Lemons } } 

 Suppose there is a market for used cars and lemons are low quality used cars,

 so there are two types of used cars, 

 - Type 1; Good quality (Peaches)

 -Type 2; Lemons

 \begin{align*}
    \begin{array}{|l|l|l|}
    \hline & \text{ Sellers } & \text { Buyers } \\
    \hline \text{Good Quality} & 2000 & 2400 \\
    \hline \text{Lemons} & 1000 & 1200\\
    \hline \text{average valuation} & 1500 & 1800\\
    \hline
    \end{array}
    \end{align*}

Suppose that both buyers and sellers know the quality or the type of used cars, then we can conclude that if there is a good quality car than any price \(p\) such that \(2000 \leq p \leq 2400\)  will be the market clearing price if there are equal number of buyers and sellers. but if there are less sellers of good qualtiy cars say 5 and more say 10 buyers of good quality cars then \(p=2400\) and if ther are more sellers \(p=2000\).  

Or we can think of the situation where neither buyers nor sellers know the quality of cars. Here all the cars will be sold.


 \subsubsection*{ Asymmetry of information } 
Now we assume that sellers know the quality of cars but buyers do not know the quality of the cars but they are aware of the fact that sellers know about the quality of the cars, and also assume that half of the cars are good quality and half of them are lemons. Then notice that on average the buyers are willing to pay at max is \(1800\) and sellers will not sell their good quality cars at all since \(2000<1800\)  and only lemons will be sold and buyers are aware of this and so they will revise their maximum willingness to pay to \(1200\),  and at this price only half of the cars i.e, only lemons will be sold and therofore asymmetry leads to market failure.



\end{document}