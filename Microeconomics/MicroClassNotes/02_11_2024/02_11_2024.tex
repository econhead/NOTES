\documentclass[12pt,a4paper]{article}

\usepackage[utf8]{inputenc}
\usepackage[T1]{fontenc}
\usepackage{parskip}
\usepackage{amsmath, amssymb, graphicx}
\usepackage{tcolorbox}
\usepackage{fancyhdr}
\setlength{\headheight}{15.6pt}
\pagestyle{fancyplain}
\fancyhead[L]{Laxman Singh}
\fancyhead[R]{\today}
\usepackage{float}
\floatstyle{boxed}
\restylefloat{figure}

\author{Laxman Singh}
\date{\today}
\title{International Trade}

\begin{document}
 \section*{Two Country Model}   
 There are two countries, Country 1 and Country 2, and a single input labor\(l\), and there are two goods, good \(X\) and good \(Y\) which are produced in both countries separately using their own labor input using the technology 
 \begin{align*}
    x_{i}&=f_{i}^X(l_{i}^X)\\
    y_{i}&=f_{i}^Y(l_{i}^Y) \quad \forall i \in \{1,2\} \\
\end{align*} 
 
 There is an international market for $X$ and $Y$ where \(X\) and \(Y\) can be traded. The are single prices for these two in the whole world \(p_{X}\) \(p_{Y}\). There are separate labor markets in both countries and the labor can not work in each other's countries, threfore there are also wage rates \(w_{1}\) and \(w_{2}\) in country 1 and country 2 respectively.
 
 There is also an autarchy case in which the goods as well as the labor can not be traded and therefore it is like the standard Crusore economy type case.

\subsubsection*{ The competitive equilibrium with internation trade}

 The competitive equilibrium with internation trade consists of \((p_{X}^*,p_{Y}^*,\omega_{1}^*,\omega_{2}^*) \in \mathbb{R}^4_{+}\) and \(\left( l_{1}^{X^*},l_{1}^{Y^*} \right),\left( x_{1}^*,y_{1}^* \right),\left( x_{1}^{c^*},y_{1}^{c^*}\right)\), \(\left( l_{2}^{X^*},l_{2}^{Y^*} \right),\left( x_{2}^*,y_{2}^* \right),\left( x_{2}^{c^*},y_{2}^{c^*}\right)\), such that,
 \begin{enumerate}
\item 
\begin{equation*}
     (l_{i}^{X^*},x_{i}^*) \text{solves} \max_{l_{i}^X,x_{i}} p_{X}^*x_{i}-\omega_{i}^*l_{i}^X \text{s.t.} x_{i}\leq f_{i}^X(l_{i}^X), \text{Let} \pi_{i}^*=p_{X}^*x_{i}^* - \omega_{i}^*l_{i}^{X^*} \forall i \in \{1,2\}
 \end{equation*}
 \item
 \begin{equation*}
    (l_{i}^{Y^*},y_{i}^*)\ \text{solves} \ \max_{l_{i}^Y,y_{i}} p_{Y}^*y_{i}-\omega_{i}^*l_{i}^Y \ \text{s.t.} \ x_{i}\leq f_{i}^Y(l_{i}^Y), \ \text{Let} \ \pi_{i}^*=p_{Y}^*y_{i}^* - \omega_{i}^*l_{i}^{Y^*} \ \forall i \in \{1,2\}
\end{equation*}
\item
\[
(x_{i}^{c^*},y_{i}^{c^*}) \ \text{solves} \ \max_{x_{i}^c,y_{i}^c} u_{i}(x_{i}^c,y_{i}^c) \ \text{s.t.} \ 
\]
\end{enumerate}


\end{document}