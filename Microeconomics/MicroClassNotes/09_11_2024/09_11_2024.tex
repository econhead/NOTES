\documentclass[12pt,a4paper,fleqn]{article}

\usepackage[utf8]{inputenc}
\usepackage[T1]{fontenc}
\usepackage{parskip}
\usepackage{amsmath, amssymb, graphicx}
\usepackage{tcolorbox}
\usepackage{fancyhdr}
\setlength{\headheight}{15.6pt}
\pagestyle{fancyplain}
\fancyhead[L]{Laxman Singh}
\fancyhead[R]{\today}
\usepackage{float}
\floatstyle{boxed}
\restylefloat{figure}
\graphicspath{{/Users/econhead/NOTES/Microeconomics/MicroClassNotes/09_11_2024/Figures}}
\author{Laxman Singh}
\date{\today}
\title{Bayesian Games}
\begin{document}
\section{\underline{Bayesian Games}}
\subsection{Swap/Retain Game}

Consider the following two-player game.
\begin{itemize}
\item The players simultaneously and independently draw a number from the set \(\{10,20,30,40,50\}\).

\item After observing the value of her own sample, which is private information (that is, opponent does not observe it), players simultaneously and independently choose one of the following: 
SWAP,RETAIN.

\item If both the players choose SWAP then they exchange their initially drawn numbers. Otherwise, if at least one person chooses RETAIN, both of them retain their numbers.

\item A player earns as many Rupees as the number she is holding at the end of the game.
\end{itemize}
 
The Choice in this game will be a strategy s, such that;
\begin{equation*}
    s:\{10,20,30,40,50\} \rightarrow \{\text{SWAP,RETAIN}\}
\end{equation*}    

Note that there are \(2^5\) number of such strategies for each of the players, and therefore there are \((2^5)^2=32^2\) number of ways in which this game can be played assuming there are no mixed strategies that can be played!

Now we can choose the strategies in this way;
 \begin{align*}
    s(50)= \text{RETAIN} & \qquad s_{\text{other}}(50)=\text{RETAIN}\\
    s(40)= \text{RETAIN} & \qquad s_{\text{other}}(40)=\text{RETAIN}\\
    s(30)= \text{RETAIN} & \qquad s_{\text{other}}(30)=\text{RETAIN}\\
    s(20)= \text{RETAIN} & \qquad s_{\text{other}}(20)=\text{RETAIN}\\
    s(10)= \text{SWAP,RETAIN} & \qquad s_{\text{other}}(10)=\text{SWAP,RETAIN}\\
\end{align*}

Note that there are four Nash Equilibriums in this game!

More Formally, 

A Bayesian Game consists of
\begin{itemize}

\item Set of Players: \(N=\{1,2, \ldots, n\}\)

\item Action Sets: Action set of player \(i\) is denoted by \(A_i\). Set of all action profiles: \(A=A_1 \times \cdots \times A_n\)

\item Type Sets: Type set of player \(i\) is denoted by \(\Theta_i\). Set of all type profiles: \(\Theta=\Theta_1 \times \cdots \times \Theta_n\)

\item Utility: Utility function of player \(i\) is: \(u_i: A \times \Theta \rightarrow \mathbb{R}\)

\item Belief: \(p_i\left(\theta_{-i} \mid \theta_i\right)\) is the conditional probability that \(i\) gives to others' types being \(\theta_{-i}\) given \(\theta_i\)
\end{itemize}


Now we can formally define the Swap/Retain game;

\begin{itemize}
    \item \(N=\{1,2\}\)  
    \item Action Sets: 
    \begin{flalign*}&A_{1}=\{\text{SWAP,RETAIN}\}&\\
    &A_{2}=\{\text{SWAP,RETAIN}\}&\\
    &A=A_{1}\times A_{2}&
    \end{flalign*}
    \item Type Sets:
    \begin{flalign*} 
     &\Theta_{1}=\{10,20,30,40,50\}&\\
     &\Theta_{2}=\{10,20,30,40,50\}&\\
     &\Theta = \Theta_{1} \times \Theta_{2}&
    \end{flalign*}
    \item Utility: \(u_{i}A_{1}\times A_{2} \times \Theta_{1} \times \Theta_{2} \to \mathbb{R}\) is defined as follows
     \begin{flalign*}
        &u_{1}(a_{1},a_{2},\theta_{1},\theta_{2})=
        \begin{cases}
            \theta_{2} & \text{if} \ a_{1}=a_{2}=\text{SWAP}\\
            \theta_{1} & \text{otherwise}
        \end{cases}&\\
        &u_{2}(a_{1},a_{2},\theta_{1},\theta_{2})=
        \begin{cases}
            \theta_{1} & \text{if} \ a_{1}=a_{2}=\text{SWAP}\\
            \theta_{2} & \text{otherwise}
        \end{cases}&
    \end{flalign*}
    \item \(p_{1}(\theta_{2}|\theta_{1})=\frac{1}{5} \qquad \forall \theta_{2} \in \Theta_{2}, \quad \forall \theta_{1} \in \Theta_{1}\) 
    
    \(p_{2}(\theta_{1}|\theta_{2})=\frac{1}{5} \qquad \forall \theta_{1} \in \Theta_{1}, \quad \forall \theta_{2} \in \Theta_{2}\)  
\end{itemize}
Both players know the joint distribution of their types!
\paragraph{}
Now, Strategy of Player \(i\) in a Bayesian Game is a function from type set of the player \(\Theta_{i}\); to the set fo actions \(A_{i}\)   he can take.
\begin{align*}
s_{i}: \theta_i \rightarrow A_i
\end{align*}
\(s\left(\theta_i\right) \in A_1\) is the action specified by strategy s; for type
\(\hat{\theta}_i \in \Theta_i\), i.e. if \(i\) plays according to s, then he takes an action \(s_5\left(\theta_1\right)\) when his type his \(\theta_i\). Let \(S_i\) denotes set of all such functions.

\begin{itemize}
    \item A strategy profile is given by
    \begin{align*}
    s=\left(s_1, s_2, \ldots, s_n\right) \in S_1 \times \cdots \times S_n
    \end{align*}
    
    \item Let us call the set of all strategy profiles \(S\) i.e.
    \begin{align*}
    S=S_1 \times \cdots \times S_n
    \end{align*}
    
    \item Expected utility is a function \(U_i: A_{i} \times S_{-i} \times \Theta_i \rightarrow \mathbb{R}\).
    \item Expected utility of type \(\theta_i\) of player \(i\) when players play according to \(s\) is given by
    
    \begin{align*}
    U_i\left(s_i\left(\theta_i\right), s_{-i}, \theta_i\right)=\sum_{\theta_{-i} \in \Theta_{-i}} p_i\left(\theta_{-i} \mid \theta_i\right) u_i\left(s_i\left(\theta_i\right), s_{-i}\left(\theta_{-i}\right), \theta_i, \theta_{-i}\right)
    \end{align*}
\end{itemize}

Bayesian Nash equilibrium
\begin{itemize}
\item Strategy profile \(s^*=\left(s_1^*, s_2^*, \ldots, s_n^*\right) \in S\) constitutes the Bayesian Nash equilibrium if for all \(i\), for all \(\theta_i, s_i^*\left(\theta_i\right)\) maximizes player \(i\) 's expected utility given that the other players play according to \(s_{-i}^*\) i.e. \(s^*\) is the Bayesian Nash equilibrium if \(\forall i \in N, \forall \theta_i \in \Theta_i\),

\begin{align*}
U_i\left(s_i^*\left(\theta_i\right), s_{-i}^*, \theta_i\right) \geq U_i\left(a_i, s_{-i}^*, \theta_i\right) \quad \forall a_i \in A_i
\end{align*}
\end{itemize}

Suppose 

\(s_{2}(10)=s_{2}(20)=\text{SWAP}\)  

\(s_{2}(30)=s_{2}(40)=s_{2}(50)=\text{RETAIN}\)  

Find the Best Response strategy for player 1?
 \begin{align*}
    &u_{1}(\text{Retain}, \ s_{2}, 50) = 50\\
    &u_{1}(\text{Swap}, \ s_{2}, 50) = \frac{1}{5} \times 10 + \frac{1}{5} \times 20 + \frac{3}{5} \times 50 = \frac{180}{5} = 36
\end{align*}
\begin{align*}
    &u_{1}(\text{Retain}, \ s_{2}, 40) = 40\\
    &u_{1}(\text{Swap}, \ s_{2}, 40) = \frac{1}{5} \times 10 + \frac{1}{5} \times 20 + \frac{3}{5} \times 40 = \frac{150}{5} = 30
\end{align*}
\begin{align*}
    &u_{1}(\text{Retain}, \ s_{2}, 30) = 30\\
    &u_{1}(\text{Swap}, \ s_{2}, 30) = \frac{1}{5} \times 10 + \frac{1}{5} \times 20 + \frac{3}{5} \times 30 = \frac{120}{5} = 24
\end{align*}
\begin{align*}
    &u_{1}(\text{Retain}, \ s_{2}, 20) = 20\\
    &u_{1}(\text{Swap}, \ s_{2}, 20) = \frac{1}{5} \times 10 + \frac{1}{5} \times 20 + \frac{3}{5} \times 20 = \frac{90}{5} = 18
\end{align*}
\begin{align*}
    &u_{1}(\text{Retain}, \ s_{2}, 10) = 10\\
    &u_{1}(\text{Swap}, \ s_{2}, 10) = \frac{1}{5} \times 10 + \frac{1}{5} \times 20 + \frac{3}{5} \times 10 = \frac{60}{5} = 12
\end{align*}
So we get the Best Response strategy of player 1 as;
 \begin{align*}
    s_{1}(50)=\text{RETAIN}\\
    s_{1}(40)=\text{RETAIN}\\
    s_{1}(30)=\text{RETAIN}\\
    s_{1}(20)=\text{RETAIN}\\
    s_{1}(10)=\text{SWAP}\\
\end{align*}

Given this best response of player 1, player 2 will also fix his own

\subsection{Battle of Sexes}
\begin{itemize}

\item There are two possible types of player 2 (column): "Meet" player 2 wishes to be at the same place as player 1, just as in the usual game (This type has probability \(p=1 / 4\) ). "Avoid" 2 wishes to avoid player 1 and go to the other place (This type has probability \(1-p=3 / 4\) ).
\item 2 knows his type, and 1 does not. They simultaneously choose the place Movie \((M)\) or Shopping \((S)\). These payoffs are shown in the matrices below.
\end{itemize}

\begin{figure}[H]
    \centering
    \includegraphics[scale=0.8]{1.png}
\end{figure}

 \begin{align*}
    &N=\{1,2\}\\
    &A_{1}=\{S,M\}\\
    &A_{2}=\{S,M\}\\
    &\Theta_{1}=\{\text{Meet}\} \quad \Theta_{2}=\{\text{Meet, Avoild}\}\\
    &u_{1}: A_{1}\times A_{2}\times \Theta_{1} \times \Theta_{2} \rightarrow \mathbb{R}\\
    &u_{1}(S,S,\text{Meet,Meet})=2\\
    &u_{1}(S,S,\text{Meet,Avoid})=2\\
    &P_{2}(\theta_{1} \mid \theta_{2})= 1 \ \text{for} \ \theta_{1}= \ \text{Meet} \ \forall \theta_{2}\\
    &P_{1}(\theta_{2} \mid \theta_{1}) = \begin{cases} \frac{1}{4} & \ \text{for} \ \theta_{2}= \ \text{Meet} \\
    \frac{3}{4} & \ \text{for} \ \theta_{2}= \ \text{Avoid} \\
 \end{cases}
\end{align*}

 \begin{align*}
    &s_{1}=S \\
    &s_{2}(\text{Meet})=S \\
    &s_{2}(\text{Avoid})=M\\
    &u_{1}(S,s_{2},\text{Meet})=2\left( \frac{1}{4} \right) + 0 \left( \frac{3}{4} \right) = \frac{1}{2}\\
    &u_{1}(M,s_{2},\text{Meet})=0\left( \frac{1}{4} \right) + 1 \left( \frac{3}{4} \right) = \frac{3}{4}\\
\end{align*}
Now
\begin{align*}
    &s_{1}=M \\
    &s_{2}^{'}(\text{Meet})=M \\
    &s_{2}^{'}(\text{Avoid})=S\\  
    &u_{1}(S,s_{2}^{'},\text{Meet})=0\left( \frac{1}{4} \right) + 2 \left( \frac{3}{4} \right) = \frac{3}{2}\\
    &u_{1}(M,s_{2}^{'},\text{Meet})=1\left( \frac{1}{4} \right) + 0 \left( \frac{3}{4} \right) = \frac{1}{4} \\
\end{align*}

\(M \to (M,S) \to S \to (S,M) \to M\) 

Therefore there is no Bayesian Nash Equilibrium in Pure strategies of this game. 

\end{document}