\documentclass[12pt,a4paper]{article}

\usepackage[utf8]{inputenc}
\usepackage[T1]{fontenc}
\usepackage{parskip}
\usepackage{amsmath, amssymb, graphicx}
\usepackage{tcolorbox}
\usepackage{fancyhdr}
\setlength{\headheight}{15.6pt}
\pagestyle{fancyplain}
\fancyhead[L]{Laxman Singh}
\fancyhead[R]{\today}
\usepackage{float}
\floatstyle{boxed}
\restylefloat{figure}
\graphicspath{{/Users/econhead/Econschool-notes/Mathematics/Convexity/Figures}}
\author{Laxman Singh}
\date{\today}
\title{Convexity}

\begin{document}
\section{Convexity/Concavity of single variable functions} 
If we arbitrariliy pick any two points on the graph of a function and connect those two points
 with a line segment, if no part of this line segment lies above the graph of the function then
  this function is concave and if no part of this line segment lies below the graph of the 
  function then the function is said to be convex.

examples of concave functions are \(f(x)=\max(x,1)\), \(f(x)=x+1\), 

\(f(x)=-x^2\), etc. 

examples of convex functions are \(f(x)=x^2\), \(f(x)=x+1\), etc.

Note: if we have a twice differentiable function then we can use the second derivatrive test
 for checking the concavity of the function.

Can there be a concave function which is discontinous?

If we have a function defined on an open interval of the real line then we can never find a
discontinous concave function.

But we have a function defined on a closed interval of the real line then we can find examples
of discontinous concave function.

\section{Convexity/Concavity of multi variable functions}
\(f(x,y)=x^{0.5}y^{0.5}\) is a concave function but it's level curves are convex.

\(f(x,y)=xy\) is neither concave nor convex but it's level curves are convex.

\(f(x,y)=\min(x,y)\) is a concave function but it's level curves are convex. 

\(f(x,y)= x+2y\) is both concave and convex and it's level curves are both concave and convex
as well.

but \(f(x,y)= (x+2y)^2\) is a convex function and it's level curves are straight lines.

\(u(x,y)=\max(\min(x,2y),\min(2x,y))\) is neither convex nor concave and some holds for it's
level curves.

Note: Level curves are not functions. 
\section{Convex Sets} 
 \subsection{Convex Combination} 
  Given two vectors \(x,x' \in \mathbb{R}\) and \(\lambda \in [0,1]\), a vector \(\lambda x +
  (1-\lambda)x'\) is known as the convex combination of \(x\) and \(x'\).      
 
  so set of all convex combinations of \(x\) and \(x'\) is the set of all the points lying on
  the line segment joining \(x\) and \(x'\).
  
  \begin{figure}[ht]
      \centering
      \includegraphics[scale=0.36]{1.png}
  \end{figure}
  
  A set \(S \subset \mathcal{R}^n\) is convex if \(\lambda x + (1-\lambda)x' \in S\) whenever 
  \(x \in S\), \(x' \in S\), and \(\lambda \in [0,1]\).
  
  \begin{figure}[ht]
      \centering
      \includegraphics[scale=0.36]{2.png}
  \end{figure}
  \pagebreak

  Can we say anything about Union and Intersection of two Convex sets?
  \begin{figure}[H]
      \centering
      \includegraphics[scale=0.35]{3.png}
  \end{figure}

  Note that Intersection of two convex sets is alaways a convex set but their union is not
  necessarily convex.

  \textbf{\underline{Proof:}} 
  \begin{itemize}
    \item Let \(X\) adn \(Y\) be convex sets, Pick arbitray \(a\) and \(b\) from the set \(X 
    \cap Y\).
    \item Notice that \(a,b \in X\) and \(a,b \in Y\)
    \item Consider any \(\lambda \in [0,1]\). Since \(X\) and \(Y\) are convex sets, we have \(\lambda a + (1-\lambda)b \in X\) and \(\lambda a + (1-\lambda)b \in Y\).
    \item therefore \(\lambda a + (1-\lambda)b \in X \cap Y\).  
  \end{itemize}

   \subsection{Concave Functions} 
   Let \(f: S \rightarrow \mathcal{R}\) be a function defined on the convex set \(S \subset \mathcal{R}^n\). Then \(f\) is concave on the set \(S\) if for all \(x \in S\), all \(x^{\prime} \in S\), and all \(\lambda \in(0,1)\) we have

\begin{align*}
f\left(\lambda x+(1-\lambda) x^{\prime}\right) \geq \lambda f(x)+(1-\lambda) f\left(x^{\prime}\right)
\end{align*}

\begin{figure}[H]
    \centering
    \includegraphics[scale=0.4]{4.png}
\end{figure}

 \subsection{Convex Funtions}
 Let \(f: S \rightarrow \mathcal{R}\) be a function defined on the convex set \(S \subset \mathcal{R}^n\). Then \(f\) is convex on the set \(S\) if for all \(x \in S\), all \(x^{\prime} \in S\), and all \(\lambda \in(0,1)\) we have

\begin{align*}
f\left(\lambda x+(1-\lambda) x^{\prime}\right) \leq \lambda f(x)+(1-\lambda) f\left(x^{\prime}\right)
\end{align*}
 
\begin{figure}[H]
    \centering
    \includegraphics[scale=0.4]{5.png}
\end{figure}

 \subsection{Sum Theorem} 
 Sum of two concave functions is a concave function:

 If \(f: S \rightarrow \mathcal{R}\) and \(g: S \rightarrow \mathcal{R}\) are two concave functions, defined on the convex set \(S \subset \mathcal{R}^n\) then
  \(t: S \rightarrow \mathcal{R}\) defined as
 \begin{align*}
  t(x)=f(x)+g(x)
 \end{align*}
 will be a concave function.

 \underline{\textbf{Proof:}} 
 \begin{itemize}
    \item Pick arbitrary \(x, x^{\prime} \in S\) and \(\lambda \in[0,1]\)
    \begin{align*}
    \begin{array}{ll} 
    & t\left(\lambda x+(1-\lambda) x^{\prime}\right) \\
    =\quad & f\left(\lambda x+(1-\lambda) x^{\prime}\right)+g\left(\lambda x+(1-\lambda) x^{\prime}\right) \qquad \qquad \quad {[\text { By definition of } t]}\\
    \geq \quad & \lambda f(x)+(1-\lambda) f\left(x^{\prime}\right)+\lambda g(x)+(1-\lambda) g\left(x^{\prime}\right) \qquad {[\text { By concavity of } f \ \text{and} \ g]} \\
    = & \lambda(f(x)+g(x))+(1-\lambda)\left(f\left(x^{\prime}\right)+g\left(x^{\prime}\right)\right) \\
    = & \lambda t(x)+(1-\lambda) t\left(x^{\prime}\right)
    \end{array}
   \end{align*}
    
    \item Therefore, \(t\) is a concave function
 \end{itemize}

 A smiliar result holds for the sum of two convex functions which tells us that the sum of two convex functions will be a convex function.

\subsection{Concave function and hypograph is a convex set} 
Let \(f: S \rightarrow \mathbb{R}\) be a function defined on the convex set \(S \subset \mathcal{R}^n\). Then \(f\) is concave if and only if the set of points below its graph (hypograph) is convex:
\begin{tcolorbox}
    \begin{equation*}
        f \text{ is concave } \Leftrightarrow \{(x,y) \in S \times \mathcal{R} : x \in S \text{ and } y \leq f(x)\} \text{ is convex }
    \end{equation*} 
\end{tcolorbox}   
\begin{figure}[H]
    \centering
    \includegraphics[scale=0.3]{6.png}
\end{figure}
To Show: \(f\) is concave
\(\Longrightarrow\{(x, y) \in S \times \mathcal{R}: x \in S\) and \(y \leq f(x)\}\)
is a convex set
\begin{enumerate}
    \item Suppose \(f\) is concave.
    \item Pick arbitrary \(\left(x^{\prime}, y^{\prime}\right)\) and \(\left(x^{\prime \prime}, y^{\prime \prime}\right)\) from the hypograph and \(\lambda \in(0,1)\)
    \item Using (1), we get \(f\left(\lambda x^{\prime}+(1-\lambda) x^{\prime \prime}\right) \geq \lambda f\left(x^{\prime}\right)+(1-\lambda) f\left(x^{\prime \prime}\right)\)
    \item Using (2), we get \(f\left(x^{\prime}\right) \geq y^{\prime}\) and \(f\left(x^{\prime \prime}\right) \geq y^{\prime \prime}\)
    \item Using (3) and (4), we get \(f\left(\lambda x^{\prime}+(1-\lambda) x^{\prime \prime}\right) \geq \lambda y^{\prime}+(1-\lambda) y^{\prime \prime}\)
    \item Therefore, \(\left(\lambda x^{\prime}+(1-\lambda) x^{\prime \prime}, \lambda y^{\prime}+(1-\lambda) y^{\prime \prime}\right)\) belongs to the hypograph
\end{enumerate}
    
Hence, hypograph is a convex set.

To Show: \(\{(x, y) \in S \times \mathcal{R}: x \in S\) and \(y \leq f(x)\}\) is a convex set \(\Longrightarrow f\) is concave
\begin{enumerate}
    \item Suppose \(\{(x, y) \in S \times \mathcal{R}: y \leq f(x)\}\) is convex
    \item Pick any arbitrary \(x^{\prime}, x^{\prime \prime} \in S\), and any \(\lambda \in(0,1)\)
    \item We have \(\left(x^{\prime}, f\left(x^{\prime}\right)\right)\) and \(\left(x^{\prime \prime}, f\left(x^{\prime \prime}\right)\right)\) belongs to the hypograph i.e.,
    \begin{align*}
    \begin{aligned}
    & \left(x^{\prime}, f\left(x^{\prime}\right)\right) \in\{(x, y) \in S \times \mathcal{R}: y \leq f(x)\} \\
    & \left(x^{\prime \prime}, f\left(x^{\prime \prime}\right)\right) \in\{(x, y) \in S \times \mathcal{R}: y \leq f(x)\}
    \end{aligned}
    \end{align*}
    \item Using (1), we get
    \begin{align*}
    \begin{aligned}
    & \lambda\left(x^{\prime}, f\left(x^{\prime}\right)\right)+(1-\lambda)\left(x^{\prime \prime}, f\left(x^{\prime \prime}\right)\right)= \\
    & \left(\lambda x^{\prime}+(1-\lambda) x^{\prime \prime}, \lambda f\left(x^{\prime}\right)+(1-\lambda) f\left(x^{\prime \prime}\right)\right)
    \end{aligned}
    \end{align*}
    belongs to the hypograph
    \item Therefore,
    \begin{align*}
    f\left(\lambda x^{\prime}+(1-\lambda) x^{\prime \prime}\right) \geq \lambda f\left(x^{\prime}\right)+(1-\lambda) f\left(x^{\prime \prime}\right)
    \end{align*}
    \end{enumerate}
Hence, \(f\) is concave
\section{Min Theorem}
Minimum of two concave functions is a concave function:

If \(f: S \rightarrow \mathcal{R}\) and \(g: S \rightarrow \mathcal{R}\) are two concave functions, defined on the convex set \(S \subset \mathcal{R}^n\) then \(h: S \rightarrow \mathcal{R}\) defined as
\begin{align*}
h(x)=\min \{f(x), g(x)\}
\end{align*}
will be a concave function.
\begin{figure}[H]
    \centering
    \includegraphics[scale=0.35]{7.png}
\end{figure}
Hypograph of \(h(x)\) :
\begin{align*}
\{(x, y) \in S \times \mathcal{R}: x \in S \text { and } y \leq h(x)\}
\end{align*}
or, \(\{(x, y) \in S \times \mathcal{R}: x \in S\) and \(y \leq \min \{f(x), g(x)\}\}\)

or, \(\{(x, y) \in S \times \mathcal{R}: x \in S\) and \(y \leq f(x)\} \cap\{(x, y) \in S \times \mathcal{R}: x \in S\) and \(y \leq g(x)\}\)

or, Intersection of hypograph of \(f(x)\) and hypograph of \(g(x)\)

\section{Transformations of Concave functions}
If
\begin{itemize}
    \item \(f: S \rightarrow \mathcal{R}\) is concave, where \(S\) is a convex subset of \(\mathcal{R}^n\) and,
    \item \(\mathrm{g}: \mathcal{R} \rightarrow \mathcal{R}\) is increasing and concave
\end{itemize}
Then

\(g \circ f: S \rightarrow \mathbb{R}\) defined as \(g \circ f(x)=g(f(x))\) is also concave.

Pick arbitrary \(x^{\prime}, x^{\prime \prime} \in S\) and \(\lambda \in(0,1)\)
\begin{align*}
\begin{aligned}
g \circ f\left(\lambda x^{\prime}+(1-\lambda) x^{\prime \prime}\right) & =g\left(f\left(\lambda x^{\prime}+(1-\lambda) x^{\prime \prime}\right)\right) \ldots(\text { by definition of } g \circ f) \\
& \geq g\left(\lambda f\left(x^{\prime}\right)+(1-\lambda) f\left(x^{\prime \prime}\right)\right) \ldots(\because f \text { is concave and } g \text { is increasing }) \\
& \geq \lambda g\left(f\left(x^{\prime}\right)\right)+(1-\lambda) g\left(f\left(x^{\prime \prime}\right)\right) \ldots(\because g \text { is concave }) \\
& =\lambda g \circ f\left(x^{\prime}\right)+(1-\lambda) g \circ f\left(x^{\prime \prime}\right) \ldots(\text { by definition of } g \circ f)
\end{aligned}
\end{align*}
Therefore, \(g \circ f\) is a concave function.

\section{Domain Extension}
If \(f: S \rightarrow \mathcal{R}\) is a concave function, defined on the convex set \(S \subset \mathcal{R}^n\) then \(h: S \times T \rightarrow \mathcal{R}\) defined as
\begin{align*}
h(x, y)=f(x)
\end{align*}
will be a concave function, where \(T\) is a convex subset of \(\mathcal{R}\)

\underline{\textbf{Proof}}
\begin{itemize}
    \item Pick arbitrary \((x, y),\left(x^{\prime}, y^{\prime}\right) \in S \times T\) and arbitrary \(\lambda \in[0,1]\)
    \begin{align*}
    \begin{aligned}
    & h\left(\lambda x+(1-\lambda) x^{\prime}, \lambda y+(1-\lambda) y^{\prime}\right) \\
    =\quad & f\left(\lambda x+(1-\lambda) x^{\prime}\right) \qquad{[\text { By definition of } h] } \\
    \geq \quad & \lambda f(x)+(1-\lambda) f\left(x^{\prime}\right) \qquad [\text{ By concavity of  }f]\\
    = \ \ \  &\lambda h(x, y)+(1-\lambda) h\left(x^{\prime}, y^{\prime}\right) \qquad 
    [\text{ By definition of } h] 
    \end{aligned}
    \end{align*}
   \item Therefore, \(h\) is a concave function
\end{itemize}
\section{Connection b/w Concave and Convex functions}
Suppose a function \(f: S \rightarrow \mathcal{R}\) defined on the convex set \(S \subset \mathcal{R}^n\), we say \(f\) is a concave function if and only if \(-f\) is a convex function.

\underline{\textbf{Proof}}

If \(f\) is a concave function, then the following statements are equivalent; 
\begin{itemize}
    \item \(f(\lambda x + (1 -\lambda)x') \geq \lambda f(x) + (1-\lambda)f(x')\) 
    \item \(-f(\lambda x + (1 -\lambda)x') \geq -(\lambda f(x) + (1-\lambda)f(x'))\) 
    \item \(-f(\lambda x + (1 -\lambda)x') \geq \lambda (-f(x)) + (1-\lambda)(-f(x'))\)
    \item \(-f\) is a convex function.   
\end{itemize}

\subsection{Properties of a convex function:}
\begin{itemize}
    \item Sum Theorem:
    
    Sum of two convex functions is a convex function
    \item Max Theorem:
    
    Max of two convex functions is a convex function
    \item Transformation:
    
    Suppose \(f: S \rightarrow \mathcal{R}\) is convex, where \(S\) is a convex subset of \(\mathcal{R}^n\) and \(g: \mathcal{R} \rightarrow \mathcal{R}\) is increasing and convex then \(g \circ f: S \rightarrow \mathbb{R}\) defined as \(g \circ f(x)=g(f(x))\) is also a convex function
    \item Epigraph is a convex set:
    
    \(f\) is convex iff the set of points above its graph(epigraph) is a convex set
\end{itemize}
\section{Quasiconcavity}
Let \(f: S \rightarrow \mathcal{R}\) be a function defined on the convex set \(S \subset \mathcal{R}^n\). Then \(f\) is Quasiconcave if every upper contour/level set of \(f\) is convex i.e
\begin{align*}
P_a=\{x \in S: f(x) \geq a\} \text { is a convex set, } \forall  \ a \in \mathcal{R}
\end{align*}
\textbf{\underline{Examples}}
\begin{itemize}
    \item \(f: \mathcal{R} \rightarrow \mathcal{R}\) given by \(f(x)=x\)

Suppose, \(a=-2\)
\begin{align*}
\begin{gathered}
P_{-2}=\{x \in \mathcal{R}: f(x) \geq-2\} \\
P_{-2}=\{x \in \mathcal{R}: x \geq-2\} = [-2,\infty)
\end{gathered}
\end{align*}
Similiarly for any arbitrary \(a\), \(P_{a}=[a,\infty)\) which is a convex set and therefore \(f\) is a quasiconcave function.  

\item \(g: \mathcal{R}_{+}^2 \rightarrow \mathcal{R}\) given by \(g(x, y)=x y\) Suppose, \(a=4\)
\begin{align*}
&P_4=\left\{(x, y) \in \mathcal{R}_{+}^2: g(x, y) \geq 4\right\} \\
&P_4=\left\{(x, y) \in \mathcal{R}_{+}^2: x y \geq 4\right\} \text{ is a convex set }
\end{align*}
Similarly for any arbitrary \(a\), \(P_{a} = \{(x,y) \in \mathbb{R}^2_{+} : xy \geq a\}\) is a convext set and hence the function \(g\) is also quasiconcave.  

Other examples of quasiconcave functions are,
\item \(u(x,y)=\min(x,y)\) 
\item \(h(x,y)=\min(x,y) + \max(x,y) = x + y\)   
\end{itemize} 

\subsection{Averages are better than extremes}
Suppose a function \(f: S \rightarrow \mathcal{R}\) defined on the convex set \(S \subset \mathcal{R}^n\);
\begin{tcolorbox}
    \(f \text{ is quasiconcave } \Leftrightarrow \forall x, x^{\prime} \in S\), and \(\forall \lambda \in[0,1]\), 
    
    we have \(f\left(\lambda x+(1-\lambda) x^{\prime}\right) \geq \min \left\{f(x), f\left(x^{\prime}\right)\right\}\)
\end{tcolorbox}
 To Show: \(f\) is quasiconcave \( \Longrightarrow f\left(\lambda x+(1-\lambda) x^{\prime}\right) \geq \min \left\{f(x), f\left(x^{\prime}\right)\right\}\)  
\begin{enumerate}
    \item Suppose \(f\) is quasiconcave i.e \(P_a=\{y \in S: f(y) \geq a\}\) is a convex set
    \item Pick arbitrary \(x, x^{\prime} \in S\) and \(\lambda \in[0,1]\)
    \item  We know, \(f(x) \geq \min \left\{f(x), f\left(x^{\prime}\right)\right\}\) and \(f\left(x^{\prime}\right) \geq \min \left\{f(x), f\left(x^{\prime}\right)\right\}\)
    \item Using (3), we get
    \begin{align*}
    x \in P_{\min \left\{f(x), f\left(x^{\prime}\right)\right\}} \text { and } x^{\prime} \in P_{\min \left\{f(x), f\left(x^{\prime}\right)\right\}}
    \end{align*}
    \item Using (1) and (4), we get
    \begin{align*}
    \lambda x+(1-\lambda) x^{\prime} \in P_{\min \left\{f(x), f\left(x^{\prime}\right)\right\}}
    \end{align*}
    \item Using definition of \(P_a\), we get \(f\left(\lambda x+(1-\lambda) x^{\prime}\right) \geq \min \left\{f(x), f\left(x^{\prime}\right)\right\}\)
    
    Therefore, \(f\) satisfies "Averages Better than extremes" property.
\end{enumerate}
To Show: \(f\left(\lambda x+(1-\lambda) x^{\prime}\right) \geq \min \left\{f(x), f\left(x^{\prime}\right)\right\}\) \(\Longrightarrow f\) is quasiconcave i.e \(P_a\) is a convex set.
\begin{enumerate}
    \item Suppose \(f\) satisfies "Averages Better than extremes" property i.e
    \begin{align*}
    f\left(\lambda x+(1-\lambda) x^{\prime}\right) \geq \min \left\{f(x), f\left(x^{\prime}\right)\right\}
    \end{align*}
    \item Pick any arbitrary \(x, x^{\prime} \in P_a\), and arbitrary \(a \in \mathcal{R}\)
    \item Using (2), we get \(f(x) \geq a\) and \(f\left(x^{\prime}\right) \geq a\)
    \item Using (3), we get \(\min \left\{f(x), f\left(x^{\prime}\right)\right\} \geq a\)
    \item Using (1) and (4), we get
    \begin{align*}
    f\left(\lambda x+(1-\lambda) x^{\prime}\right) \geq a 
    \end{align*}
    \item Using (5) and definition of \(P_a\) we get \(\lambda x+(1-\lambda) x^{\prime} \in P_a\) i.e., \(P_a\) is a convex set.
\end{enumerate}
\subsection{Domain Restriction}
\(f: S \to \mathbb{R}\) is concave \(\Leftrightarrow f: S_{\text{ rest }} \to \mathbb{R} \) is also concave where \(S_{ \text{ rest }} \subset S\).    

If we consider any subset of \(S\) which is also convex, then restricting the function to this subset of \(S\) will give us a concave function.

\textbf{\underline{Examples}}
\begin{itemize}
    \item  Is \(u=xy\) concave?
    
    if we restrict our domain to \(x=y\) then \(u|_{\text{ rests}} = x^2\) which is not a concave function and hence \(u\) can not be concave.    
\end{itemize}
A similar result holds for convex functions.
\end{document}
