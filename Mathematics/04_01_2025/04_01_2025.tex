\documentclass[12pt,a4paper]{article}

\usepackage[utf8]{inputenc}
\usepackage[T1]{fontenc}
\usepackage{parskip}
\usepackage{amsmath, amssymb, graphicx}
\usepackage{tcolorbox}
\usepackage{fancyhdr}
\setlength{\headheight}{15.6pt}
\pagestyle{fancyplain}
\fancyhead[L]{Laxman Singh}
\fancyhead[R]{\today}
\usepackage{float}
\floatstyle{boxed}
\restylefloat{figure}
\graphicspath{{/Users/econhead/NOTES/Mathematics/04_01_2025/Figures}}       
\author{Laxman Singh}
\date{\today}
\title{Metric Spaces Continued}

\begin{document}
   \section{Metric Spaces Contd.} 
   \subsubsection{ Proposition 14} 
\begin{itemize}
    \item A set \(S\) in a metric space \(X\) is closed if, and only if, it contains all its limit points.
\end{itemize}
To prove: If A set S in a metric space \(X\) is closed then it contains all its limit points.
\begin{itemize}   
    \item Suppose \(S\) does not contain all its limit points.
    \item Then there exists \(x \in X\) which is the limit point of \(S\) but is not in \(S\).
    \item Since \(x\) is the limit point of \(S, \mathcal{N}_\epsilon(x)\setminus\{x\} \cap S \neq \emptyset\) for all \(\epsilon>0\) which implies that \(S^c\) not open and hence \(S\) is not closed.
\end{itemize}  
To prove: If A set \(S\) in a metric space contains all its limit points then it is a closed set.
\begin{itemize}
    \item Conversely, assume that \(S\) contain all its limit points but \(S\) is not closed.
    \item Then \(S^c\) is not open.
    \item Hence there is \(x \in S^c\) such that for every \(\epsilon>0, \mathcal{N}_\epsilon(x) \cap S \neq \emptyset\).
    \item This together with \(x \notin S\) implies that for every \(\epsilon>0, \mathcal{N}_\epsilon(x) \setminus \{x\} \cap S \neq \emptyset\) implying further that \(x\) is the limit point of \(S\) but \(x \notin S\), contradicting that \(S\) contains all its lim points.
\end{itemize}

To Prove; If a set \(A \subset X\) is closed in \((X,d)\) implies Set \(A\) contains all it's limit points.

The contrapositive of the above statement is, if a set \(A \subset X\) does not contain all it's limit points then the set \(A\) is not closed.       

let's prove the contrapositive
\begin{itemize}
    \item Suppose \(A\) does not contain all it's limit points.
    \item Now we want to show that \(X\setminus A\) is not open.
    \item since \(A\) does not contain all it's limit points the following must be true
    \(\exists l \in X \) such that \(l\) is the limit point of \(A \subset X\) but \(l \notin A\).
    \item but this means, \((\forall \epsilon >0)(\mathcal{N}_{\epsilon}(l)\setminus \{l\}\cap A \neq \phi\)
    \item but this implies \((\forall \epsilon >0)\mathcal{N}_{\epsilon}(l)\cap A \neq \phi\) 
    \item \(\implies (\forall \epsilon > 0)(\mathcal{N}_{\epsilon}(l) \not\subset X\setminus A\).
    \item Therefore the set \(X \setminus A\) is not open and hence set \(A\) is not closed.                  
\end{itemize}

 \subsubsection{Propositon 15} 
 \begin{itemize}
    \item A set \(S\) in a metric space \(X\) is closed if, and only if, every sequence in \(S\) that converges in \(X\) converges to a point in \(S\)
 \end{itemize}
To prove: If \(A\) set \(S\) in a metric space \(X\) is closed then every sequence in \(S\) that converges in \(X\) converges to a point in \(S\)
\begin{itemize}
    \item Assume that \(S\) is closed and that \(\left(x_n\right)\) is a sequence of points belonging to \(S\) converging to \(x \in X\).
    \item We claim that \(x \in S\).
    \item Arguing by contradiction, we assume that \(x \notin S\).
    \item Since \(S^c\) is open, there is \(r_x>0\) such that \(\mathcal{N}_{r_{\varepsilon}}(x) \subseteq S^c\).
    \item Then, since \(x_n \in S, d\left(x_n, x\right) \geq r_x\), contradicting that \(d\left(x_n, x\right) \rightarrow 0\).
\end{itemize}
To prove: If every sequence in \(S\) that converges in \(X\) converges to a point in \(S\) then \(S\) in a metric space \(X\) is closed
\begin{itemize}
    \item Conversely, assume that every convergent sequence ( \(x_n\) ) such that \(x_n \in S\) converges to a point in \(S\) but \(S\) is not closed. Then \(S^c\) is not open.
    \item Hence there is \(x \in S^c\) such that for every \(r>0, \mathcal{N}_r(x) \cap S \neq \emptyset\).
    \item In particular, for numbers \(1 / n\) where \(n \geq 1\), we find points \(x_n \in S\) such that \(d\left(x_n, x\right)<1 / n\).
    \item Hence the sequence ( \(x_n\) ) converges to \(x\) and \(x \notin S\), contradiction.
\end{itemize}

To Prove; A set \(A \subset X\) is closed or it contains all it's limit points if \((\forall(x_{n}) \subset A)(\text{if there is} \ l \in X \ \text{such that} \ x_{n} \to l \implies l \in A)\).
\begin{itemize}
    \item The contrapositive of the above statement is; If a set does not contain all it's limit points then it contains a sequence that converges to a point that does not lie in the set or lies outside of it.
    \item Suppose set \(A\) does not contain all it's limit points
    \item \(\implies (\exists l \in X\setminus A)(\forall \epsilon >0)(\mathcal{N}_{\epsilon}(l)\setminus \{l\}\cap A \neq \phi\)
    \item \(\implies (\exists l \in X\setminus A)(\forall n \in \mathbb{N})(\mathcal{N}_{\frac{1}{n}}(l)\setminus \{l\}\cap A \neq \phi\)
    \item now if we pick a point form this non-empty set \(x_{n} \in \mathcal{N}_{\frac{1}{n}}\setminus \{l\}\cap A\)
    \item Noitce that \(0\leq d(x_{n},l)<\frac{1}{n} \quad \forall n \in \mathbb{N}\) is true and also \(x_{n} \in A\)           
    \item so by squeeze theorem \(d(x_{n},l)\to 0\) and \(l \in X \setminus A\)    
\end{itemize}
 \subsubsection{Propositon 16} 
\begin{itemize}
    \item Let \(S\) be a nonempty bounded subset of \(\mathbb{R}\). Show that there is an increasing sequence \(\left(x_n\right)\) in \(S\) such that \(x_n \rightarrow \sup S\) and a decreasing sequence \(\left(y_n\right)\) in \(S\) such that \(y_n \rightarrow \inf S\).
\end{itemize}
To prove: There is an increasing sequence \(\left(x_n\right)\) in \(S\) such that \(x_n \rightarrow \sup S\)
\begin{itemize}
    \item - Let \(S\) be a nonempty bounded subset of \(\mathbb{R}\) and \(x=\sup S\).
    - For numbers \(1 / n\) where \(n \geq 1\), we find points \(x_n \in S\) such that \(d\left(x_n, x\right)<1 / n\).
    - Hence, the sequence ( \(x_n\) ) converges to \(x\).
    - Consider the sequence \(\left(y_n\right)\) obtained from \(\left(x_n\right)\) in the following way:
\begin{align*}
y_n=\max \left\{x_1, x_2, \ldots, x_n\right\}
\end{align*}
\item Clearly, \(\left(y_n\right)\) is an increasing sequence that converges to \(x\).
\item The proof of the second claim is analogous. 
\end{itemize}

Let \(A\)   be an non-empty bounded subset of \(\mathbb{R}\), this means it has finite infimum and supremum, and \(A\) contains an increasing sequence that converges to \(\sup A\) and \(A\) also contains a decreasing sequence that converges to \(\inf A\).     

let's prove the first part that it contains an increasing sequence...

There are two possiblities that is 
\begin{itemize}
    \item if \(\sup A \in A \), then \(x_{n}=\sup\) is the required sequence since a constant sequence is an increasing sequence.
    \item if \(\sup A \notin A\), then pick \(x_{1} \in (\sup A -1, \sup A) \cap A\) and let \(d_{2}=\min(\frac{1}{2},|x_{1}-\sup A|\)
    \item then we pick \(x_{2} \in \sup A -d_{2}, \sup A) \cap A\) and let \(d_{3}=\min(\frac{1}{3},|x_{2}-\sup A|\), and so on pick \(x_{3} \in (\sup A -d_{3}, \sup A) \cap A\),...
    \item note that since \(d_{n} \to 0\) the aboslute value of the distance between \(x_{n}\) and \(\sup A\) goes to \(0\).
    \item therefore we get an increasing sequence in this manner that converges to \(\sup A\).                        
\end{itemize}

 \subsection{Compact Sets and Totally Bounded Sets}
  \subsubsection{Compact Set} 
\begin{itemize}
    \item - Let \((X, d)\) be a metric space. A subset \(S\) of a metric space \(X\) is compact if every sequence in \(S\) has a subsequence that converges to a point in \(S\).
\end{itemize}
Exapmles;
\begin{itemize}
    \item Is \(A_{1}=\mathbb{R}\) a compact set? No! there are numerous sequences in \(\mathbb{R}\) which have non convergent subsequences.
    \item Is \(A_{2}=\phi\) a compact set? Yes, vacously true!   
    \item Is \(A_{3}=(0,1)\) a compact set? No, since the sequence \(x_{n}=\frac{1}{n+1}\in A_{3}\) has no subsequence that converges to a point in \(A_{3}\). Basically in other words we also need a set to be closed and bounded for it to be compact in euclidean metric spaces but is not necessarily true for other metric spaces.
    \item Consider \(X=[0,1]\) and \[d(x,y)=\begin{cases} 1 & \text{if} \ x \neq y\\ 0 & \text{if} \ x=y \end{cases}\] Then is \(\frac{1}{n+1}\) a convergent sequence in \((X,d)\)? 
    
    No! because we say that \(x_{n}\to l\) in \((X,d)\) if \(d(x_{n},l) \to 0\)   
    
    A convergent sequence in discrete metric space has the property that all of it's terms except possibly finitely many terms are same, for example \(x_{n}= 1, \frac{1}{2},\frac{1}{2},\frac{1}{2},\ldots \)    
    \item is \(A=(0,1) \in (X,d)\) compact? where \(X=[0,1]\) and \(d(x,y)\) is the discrete metric? No it is not compact.
    \item is \(A_{2}=[0,1] \in (X,d)\) compact? No
    \item is \(A_{3}=\{0,1\} \in (X,d)\) compact? Yes
    \item Any finite subset of the set \(X\) will be compact.
    \item let \(X=[0,1]\) and \(d(x,y)=|x-y|\) then is \(A=[0,1] \in (X,d)\) compact?  Yes, it is easier to see that this set is closed and bounded than to find a sequence that has all it's subsequences coverging to a point in \(A\).                   
\end{itemize}

 \subsubsection{Totally Bounded Set} 
\begin{itemize}
    \item  A set \(S\) in \(X\) is said to be totally bounded (or precompact) if, for any \(\epsilon>0\), there exists a finite subset \(T\) of \(X\) such that
    \begin{align*}
    S \subseteq \bigcup_{x \in T} \mathcal{N}_\epsilon(x)
    \end{align*}
\end{itemize}
Examples;
\begin{itemize}
    \item say \(\mathbb{R},d(x,y)\) is our metric space where d(x,y) is the discrete metric, Is the set \((0,1)\) bounded in \((\mathbb{R},d)\)? yes(any epsilon greater than works for the neighbourhood) but is it totally bounded? no (it fails for \(\epsilon=\frac{1}{2}\)).
\end{itemize}
 
\subsubsection{Proposiotion 17} 
\begin{itemize}
    \item Every totally bounded subset of a metric space is bounded.
\end{itemize}
Proof:
\begin{itemize}
    \item Consider a totally bounded subset \(S\) of \(X\).
    \item Thus, \(S \subseteq \cup_{i=1}^n \mathcal{N}_1\left(t_i\right)\) for some \(T=\{ t_1, \ldots, t_n \} \subseteq X\).
    \item \(S\) is bounded because for \(r=\max \left\{d\left(t_i, t_j\right) \mid t_i, t_j \in T\right\}+1, S \subseteq \mathcal{N}_r\left(t_1\right)\).
\end{itemize}

To Prove that a totally bounded set is bounded;

Proof:
\begin{itemize}
    \item Suppoe a set \(A\)  is totally bounded.
    \item WTS; \(A\) is bounded.
    \item Note that it is vacously true that if \(A\) is empty then it is bounded. so we only need to consider a non-empty set \(A\) that is totally bounded.
    \item Since \(A\) is totally bounded, for \(\epsilon=1\), there exists finite set \(T \subset X\) such that \[A \subset \bigcup_{x\in T}\mathcal{N}_{1}(x)\] 
    \item Then we can take the \(\max d(x_{i},x_{j)}\) where \(x_{i},x_{j} \in T\) for the bigger ball or
    \item for \(x_{1} \in T\), take \(l=\max_{x_{i}\in T} d(x_{i}, x_{1})\) then we wnat to show that \(A \subset \mathcal{N}_{l+1}(x_{1})\) 
    \item we can prove this as follows;
    \begin{align*}
        &p \in A \implies p \in \mathcal{N}_{l+1}(x_{1})\\
        &d(p,x_{1}) \leq d(p,x_{k_{p}}) + d (x_{k_{p}}, x_{1}) \qquad [\text{By triangle's inequality}]
    \end{align*}
     and therefore \( p \in \mathcal{N}_{l+1}(x_{1})\) must be true.  
\end{itemize}

\end{document}
