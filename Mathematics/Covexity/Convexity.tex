\documentclass[12pt,a4paper]{article}

\usepackage[utf8]{inputenc}
\usepackage[T1]{fontenc}
\usepackage{parskip}
\usepackage{amsmath, amssymb, graphicx}
\usepackage{tcolorbox}
\usepackage{fancyhdr}
\setlength{\headheight}{15.6pt}
\pagestyle{fancyplain}
\fancyhead[L]{Laxman Singh}
\fancyhead[R]{\today}
\usepackage{float}
\floatstyle{boxed}
\restylefloat{figure}
\graphicspath{{/Users/econhead/NOTES/Mathematics/Covexity/Figures}}
\author{Laxman Singh}
\date{\today}
\title{Convexity}

\begin{document}
\section{Convexity/Concavity of single variable functions} 
If we arbitrariliy pick any two points on the graph of a function and connect those two points
 with a line segment, if no part of this line segment lies above the graph of the function then
  this function is concave and if no part of this line segment lies below the graph of the 
  function then the function is said to be convex.

examples of concave functions are \(f(x)=\max(x,1)\), \(f(x)=x+1\), 

\(f(x)=-x^2\), etc. 

examples of convex functions are \(f(x)=x^2\), \(f(x)=x+1\), etc.

Note: if we have a twice differentiable function then we can use the second derivatrive test
 for checking the concavity of the function.

Can there be a concave function which is discontinous?

If we have a function defined on an open interval of the real line then we can never find a
discontinous concave function.

But we have a function defined on a closed interval of the real line then we can find examples
of discontinous concave function.

\section{Convexity/Concavity of multi variable functions}
\(f(x,y)=x^{0.5}y^{0.5}\) is a concave function but it's level curves are convex.

\(f(x,y)=xy\) is neither concave nor convex but it's level curves are convex.

\(f(x,y)=\min(x,y)\) is a concave function but it's level curves are convex. 

\(f(x,y)= x+2y\) is both concave and convex and it's level curves are both concave and convex
as well.

but \(f(x,y)= (x+2y)^2\) is a convex function and it's level curves are straight lines.

\(u(x,y)=\max(\min(x,2y),\min(2x,y))\) is neither convex nor concave and some holds for it's
level curves.

Note: Level curves are not functions. 
\section{Convex Sets} 
 \subsection{Convex Combination} 
  Given two vectors \(x,x' \in \mathbb{R}\) and \(\lambda \in [0,1]\), a vector \(\lambda x +
  (1-\lambda)x'\) is known as the convex combination of \(x\) and \(x'\).      
 
  so set of all convex combinations of \(x\) and \(x'\) is the set of all the points lying on
  the line segment joining \(x\) and \(x'\).
  
  \begin{figure}[ht]
      \centering
      \includegraphics[scale=0.36]{1.png}
  \end{figure}
  
  A set \(S \subset \mathcal{R}^n\) is convex if \(\lambda x + (1-\lambda)x' \in S\) whenever 
  \(x \in S\), \(x' \in S\), and \(\lambda \in [0,1]\).
  
  \begin{figure}[ht]
      \centering
      \includegraphics[scale=0.36]{2.png}
  \end{figure}
  \pagebreak

  Can we say anything about Union and Intersection of two Convex sets?
  \begin{figure}[H]
      \centering
      \includegraphics[scale=0.35]{3.png}
  \end{figure}

  Note that Intersection of two convex sets is alaways a convex set but their union is not
  necessarily convex.

  \textbf{\underline{Proof:}} 
  \begin{itemize}
    \item Let \(X\) adn \(Y\) be convex sets, Pick arbitray \(a\) and \(b\) from the set \(X 
    \cap Y\).
    \item Notice that \(a,b \in X\) and \(a,b \in Y\)
    \item Consider any \(\lambda \in [0,1]\). Since \(X\) and \(Y\) are convex sets, we have \(\lambda a + (1-\lambda)b \in X\) and \(\lambda a + (1-\lambda)b \in Y\).
    \item therefore \(\lambda a + (1-\lambda)b \in X \cap Y\).  
  \end{itemize}

   \subsection{Concave Functions} 
   Let \(f: S \rightarrow \mathcal{R}\) be a function defined on the convex set \(S \subset \mathcal{R}^n\). Then \(f\) is concave on the set \(S\) if for all \(x \in S\), all \(x^{\prime} \in S\), and all \(\lambda \in(0,1)\) we have

\begin{align*}
f\left(\lambda x+(1-\lambda) x^{\prime}\right) \geq \lambda f(x)+(1-\lambda) f\left(x^{\prime}\right)
\end{align*}

\begin{figure}[H]
    \centering
    \includegraphics[scale=0.4]{4.png}
\end{figure}

 \subsection{Convex Funtions}
 Let \(f: S \rightarrow \mathcal{R}\) be a function defined on the convex set \(S \subset \mathcal{R}^n\). Then \(f\) is convex on the set \(S\) if for all \(x \in S\), all \(x^{\prime} \in S\), and all \(\lambda \in(0,1)\) we have

\begin{align*}
f\left(\lambda x+(1-\lambda) x^{\prime}\right) \leq \lambda f(x)+(1-\lambda) f\left(x^{\prime}\right)
\end{align*}
 
\begin{figure}[H]
    \centering
    \includegraphics[scale=0.4]{5.png}
\end{figure}

 \subsection{Sum Theorem} 
 Sum of two concave functions is a concave function:

 If \(f: S \rightarrow \mathcal{R}\) and \(g: S \rightarrow \mathcal{R}\) are two concave functions, defined on the convex set \(S \subset \mathcal{R}^n\) then
  \(t: S \rightarrow \mathcal{R}\) defined as
 \begin{align*}
  t(x)=f(x)+g(x)
 \end{align*}
 will be a concave function.

 \underline{\textbf{Proof:}} 
 \begin{itemize}
    \item Pick arbitrary \(x, x^{\prime} \in S\) and \(\lambda \in[0,1]\)
    \begin{align*}
    \begin{array}{ll} 
    & t\left(\lambda x+(1-\lambda) x^{\prime}\right) \\
    =\quad & f\left(\lambda x+(1-\lambda) x^{\prime}\right)+g\left(\lambda x+(1-\lambda) x^{\prime}\right) \qquad \qquad \quad {[\text { By definition of } t]}\\
    \geq \quad & \lambda f(x)+(1-\lambda) f\left(x^{\prime}\right)+\lambda g(x)+(1-\lambda) g\left(x^{\prime}\right) \qquad {[\text { By concavity of } f \ \text{and} \ g]} \\
    = & \lambda(f(x)+g(x))+(1-\lambda)\left(f\left(x^{\prime}\right)+g\left(x^{\prime}\right)\right) \\
    = & \lambda t(x)+(1-\lambda) t\left(x^{\prime}\right)
    \end{array}
   \end{align*}
    
    \item Therefore, \(t\) is a concave function
 \end{itemize}

 A smiliar result holds for the sum of two convex functions which tells us that the sum of two convex functions will be a convex function.

 
\end{document}