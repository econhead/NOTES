\documentclass[12pt,a4paper]{article}

\usepackage[utf8]{inputenc}
\usepackage[T1]{fontenc}
\usepackage{parskip}
\usepackage{amsmath, amssymb, graphicx}
\usepackage{tcolorbox}
\usepackage{fancyhdr}
\setlength{\headheight}{15.6pt}
\pagestyle{fancyplain}
\fancyhead[L]{Laxman Singh}
\fancyhead[R]{\today}
\usepackage{float}
\floatstyle{boxed}
\restylefloat{figure}
\graphicspath{{/Users/econhead/NOTES/Mathematics/MathClassNotes/10_01_2025/Figures}}
\author{Laxman Singh}
\date{\today}
\title{Compact Sests and Continuity of functions }

\begin{document}
\section{Compact Sets Contd.} 
\subsection{Proposition 18}
\begin{itemize}
    \item Any compact subset of a metric space \(X\) is closed and totally bounded.
\end{itemize}
\begin{itemize}
    \item We will first show that any compact subset of a metric space \(X\) is closed:
    \item Let \(S\) be the compact subset of \(X\).
    \item Take any \(\left(x_n\right)\) in \(S\) with \(x_n \rightarrow I_x\) for some \(I_x \in X\).
    \item We just need to show \(I_x \in S\) (See Proposition 15 ).
    \item Since \(S\) is compact there is a subsequence \(\left(x_{n_*}\right)\) of \(\left(x_n\right)\) that converges to a point in \(S\).
    \item Since \(\left(x_n\right)\) converges to \(I_x\), so \(\left(x_{n_k}\right)\) also converges to \(I_x\). Thus, \(I_x \in S\).
\end{itemize}
Now let us show that any compact subset of a metric space \(X\) is totally bounded:
\begin{itemize}
    \item Suppose the claim is not true, that is, there is a compact subset \(S\) of \(X\) with the following property: there exists an \(\epsilon>0\) such that \(\left\{\mathcal{N}_e(x): x \in T\right\}\) does not cover \(S\) for any finite \(T \subseteq S\).
    \item To derive a contradiction, we wish to construct a sequence in \(S\) with no convergent subsequence.
    \item Begin by picking an \(x_1 \in S\) arbitrarily.
    \item By hypothesis, we cannot have \(S \subseteq \mathcal{N}_{\varepsilon}\left(x_1\right)\) so there must exist an \(x_2 \in S\) such that \(d\left(x_1, x_2\right) \geq \epsilon\).
    \item Again, \(S \subseteq \mathcal{N}_\epsilon\left(x_1\right) \cup \mathcal{N}_\epsilon\left(x_2\right)\) cannot hold, so we can find an \(x_3 \in S\) such that \(d\left(x_i, x_3\right) \geq \epsilon, i=1,2\).
    \item Proceeding inductively, we obtain a sequence \(\left(x_n\right)\) such that \(d\left(x_i, x_j\right) \geq \epsilon\) for any distinct \(i, j \in \mathbb{N}\).
    \item Since \(S\) is compact, there must exist a convergent subsequence, say \(\left(x_{n_k}\right)\), of \(\left(x_n\right)\).
    \item But this is impossible since \(\lim x_{n_k}=x\) would imply that
    \begin{align*}
    d\left(x_{n_k}, x_{n_j}\right) \leq d\left(x_{n_k}, x\right)+d\left(x, x_{n_j}\right)<\epsilon
    \end{align*}
 for large enough (distinct) \(k\) and \(I\).
\end{itemize}

Let \(\left( X,d \right)\) be our metric space then we want to show that \(Y \subset X\), \(Y\) is compact, implies that \(Y\) is closed and \(Y\) is totally bounded.      

\begin{enumerate}
    \item \(Y\) is compact \(\to\) Y is closed.
    Since \(Y\) is compact, so every sequence in \(Y\) has a subsequence that converges to a point in Y.  
    \begin{tcolorbox}
        \underline{\textbf{Proof}}
        Consider an arbitray sequence \(\left( x_{n} \right) \subset Y\) that converges to 
        \(l \in X \). Since \(Y\) is compact, \(l \in Y\), and therefore \(Y\) is closed.   
    \end{tcolorbox}
    \item \(Y\) is compact \(\implies \) \(Y\) is totally bounded.  
    
    \underline{\textbf{Proof}}
    The contrapositive of the above statement is;

    \(Y \) is not totally bounded \(\implies\) \(Y\) is not compact.    

    Y is not totally bounded means \(\exists \epsilon >0\) such that
    
    \[\left( \text{for no finite subset} \ T \ \text{of} \ X, Y \subset \bigcup_{x \in T} \mathcal{N}_{\epsilon}(x) \right) \]  

    or, 

    \[ \left( \text{for any finte subset} \ T \ \text{of} \ X, Y \not\subset \bigcup_{x \in T} \mathcal{N}_{\epsilon}(x) \right) \] 
    
    \(Y\)  is not compact means that there exsits a sequence in \(Y\) that has no subsequence that converges to a point in \(Y\).
    
    Since Y in not totally bounded we know that Y is not an empty set because an empty set is totally bounded, therefore we start by picking a point \(x_{1} \in Y\) then it is true \(Y \not\subset \mathcal{N}_{\epsilon}(x_{1})\) since \(Y \) is not totally bounded, 

    Now we pick any \(x_{2} \in Y \setminus \mathcal{N}_{\epsilon}(x_{1})\), we can do this because we know the first n-ball does not contain the set \(Y\) completely, then it is true that;
    \[ Y \not\subset \mathcal{N}_{\epsilon}(x_{1}) \cup \mathcal{N}_{\epsilon}(x_{2}) \]
    similiarly we now pick \(x_{3} \in Y \setminus \mathcal{N}_{\epsilon}(x_{1}) \cup \mathcal{N}_{\epsilon}(x_{2})\) and then we know 
    \begin{equation*}
        Y \not\subset \mathcal{N}_{\epsilon}(x_{1}) \cup \mathcal{N}_{\epsilon}(x_{2}) \cup \mathcal{N}_{\epsilon}(x_{3})
    \end{equation*}    
    In this manner we get our sequence \(\left( x_{n} \right) \) which has the property 
    \begin{equation*}
        d(x_{i},x_{j}) \geq \epsilon \quad \text{for } i \neq j
    \end{equation*}    
    Now suppose that \(x_{n} \to l\)
    
    then \(d(x_{n},l) \to 0\)
    
    and \(\exists \ N \text{ s.t. } n > N, d(x_{n},l)<\frac{\epsilon}{4}\)  
    
    and by triangle's inequality \[d(x_{i},x_{j}) \leq d(x_{i},l) + d(x_{j},l) < \frac{\epsilon}{4} + \frac{\epsilon}{4} = \frac{\epsilon}{2} \ \ \text{for } i \neq j\]
    which is a contradiction!
\end{enumerate}

\subsection{Proposition 19}
\begin{itemize}
    \item Given any \(m \in \mathbb{N}\), a subset of \(\mathbb{R}^m\) (with Euclidean metric) is compact if, and only if, it is closed and bounded.
\end{itemize}
If compact then closed and bounded:
\begin{itemize}
    \item  By proposition 17 and 18, every compact set is both closed 
    
    and bounded.
\end{itemize}
If closed and bounded then compact :
\begin{itemize}
    \item Now suppose \(S\) is closed and bounded.
    \item To show \(S\) is compact, pick any sequence \(\left(x_n\right)\).
    \item Since \(\left(x_n\right)\) is bounded, there exist a convergent subsequence ( \(x_{n_k}\) ) (by proposition 13).
    \item Limit of \(\left(x_{n_k}\right)\) will lie in \(S\) since \(S\) is closed (by proposition 15).
\end{itemize}

Now let's prove that In \(\left( X,d \right)\) -  Euclidean metric space \((X \subset \mathbb{R}^n)\), if \(Y \subset X\) is closed and bounded then Y is compact. 

\underline{\textbf{Proof}}

Suppose \(Y\) is closed and bounded, consider an arbitrary sequence \(\left( x_{n} \subset Y \right) \), then by Bolzano-Weierstarass theorem, since \((x_{n})\) is bounded in \(\mathbb{R}^n\), it has a convergent subsequence \(x_{n_{k}} \to l\), since \(Y\) is closed, it follows that \(l \in Y\), therefore Y is compact.       

We can find examples of metric spaces in which a set is compact and bounded that is not compact.

\(X = \mathbb{R}\) and \(d(x,y)=\begin{cases}1 & \text{ if } x \neq y\\
    0 & \text{ if } x=y \end{cases}\) 
    
Note that in this discrete metric space \(\mathbb{R}\) closed and bounded.

say we now pick the sequence \(\left( \frac{1}{n} \right) \) or \((n)\) and it is easy to observe that neither of these sequences are not convergent and none of their subsequences convergent  to a point in the set \(X\).    

\section{Continuity of a function}
\subsection{Continuity of a function at a point:}
A function \(f: (X,d_{x}) \rightarrow (Y,d_{y})\) is said to be continuous at \(x_0 \in X\) if for every \(\epsilon>0\), there exists \(\delta>0\) such that \(f\left(\mathcal{N}_\delta\left(x_0\right)\right) \subseteq \mathcal{N}_c\left(f\left(x_0\right)\right)\) (where for any subset \(A\) of \(X\), we define \(f(A)=\{f(x): x \in A\}\) ).

We say that \(f: X \to Y\) is continuous at \(x_{0} \in X\) if \((\forall \epsilon > 0)(\exists \delta > 0)(f(\mathcal{N}_{\delta}(x_{0})) \subseteq \mathcal{N}_{\epsilon}(f(x_{0})))\)  

Examples;
\begin{enumerate}
    \item Suppose \(X = \mathbb{N}\) and \(d_{x}(m,n)=|m-n|\)
    
    \(Y = \mathbb{R}\) and \(d_{y}(a,b)=|a-b|\)   
    
    Is \(f(x)=x\) where \(f: X \to Y\) continuous?
    
    Yes! because \(\mathcal{N}_{\delta}(2)=\{2\} \quad \text{ for } \delta \leq 1\) 
    
    \item Suppose \(X = \mathbb{N}\) and \(d_{x}(m,n)=|m-n|\)
    
    \(Y = \mathbb{R}\) and \(d_{y}(a,b)=\begin{cases} 1 & \text{ if } a \neq b\\ 0 & \text{ if } a=b \end{cases}\)   
    
    Is \(f(x)=x\) where \(f: X \to Y\) continuous? Yes in this case  
    
    \(f\left(\mathcal{N}_\delta\left(x_0\right)\right) = \mathcal{N}_c\left(f\left(x_0\right)\right)\)
    
    \item Suppose \(X = \mathbb{R}\) and \(d_{x}(m,n)=|m-n|\)
    
    \(Y = \mathbb{R}\) and \(d_{y}(a,b)=\begin{cases} 1 & \text{ if } a \neq b\\ 0 & \text{ if } a=b \end{cases}\)   
    
    Is \(f(x)=x\) where \(f: X \to Y\) continuous at \(x=2\)? NO!
    
    \item Suppose \(X = \mathbb{R}\) and \(d_{x}(a,b)=\begin{cases} 1 & \text{ if } a \neq b\\ 0 & \text{ if } a=b \end{cases}\)   
    
    \(Y = \mathbb{R}\) and \(d_{y}(a,b)=|a-b| \)  
    
    Is \(f(x)=\lfloor x \rfloor \) where \(f: X \to Y\) continuous at \(x=2\)? Yes! take \(\delta \leq 1\).   

\end{enumerate}
\subsection{Proposition 20}
\begin{itemize}
    \item A function \(f: X \rightarrow Y\) is continuous at \(x_0 \in X\) if, and only if, for every sequence \(\left(x_n\right)\) in \(X\) that converges to \(x_0\), the sequence \(f\left(x_n\right)\) in \(Y\) converges to \(f\left(x_0\right)\).
\end{itemize}
\(f\) is continous at \(x_{0} \in X\) iff    
\(\left( \forall (x_{n}) \subset X \right)\left( x_{n} \to x_{0} \in X \implies f(x_{n}) \to f(x_{0}) \right)  \)  

To prove : If \(f\) is continuous at \(x_0\) then for every sequence \(\left(x_n\right)\) in \(X\) that converges to \(x_0\), we have the sequence \(f\left(x_n\right)\) in \(Y\) converges to \(f\left(x_0\right)\) :
\begin{itemize}
    \item Suppose \(f: X \rightarrow Y\) is continuous at \(x_0 \in X\).
    \item Now pick any sequence \(\left(x_n\right)\) in \(X\) that converges to \(x_0\).
    \item We claim that \(f\left(x_n\right)\) converges to \(f\left(x_0\right)\).
    \item To prove this, consider an arbitrary \(\epsilon>0\).
    \item Since \(f\) is continuous at \(x_0\), there exists \(\delta>0\) such that \(f\left(\mathcal{N}_\delta\left(x_0\right)\right) \subseteq \mathcal{N}_\epsilon\left(f\left(x_0\right)\right)\).
    \item Now ( \(x_n\) ) converges to \(x_0\), thus there exists \(N\) such that \(x_n \in \mathcal{N}_\delta\left(x_0\right)\) for all \(n \geq N\).
    \item So, we have \(f\left(x_n\right) \in f\left(\mathcal{N}_\delta\left(x_0\right)\right) \subseteq \mathcal{N}_\epsilon\left(f\left(x_0\right)\right)\) for all \(n \geq N\).
    \item Hence, \(f\left(x_n\right)\) converges to \(f\left(x_0\right)\).
\end{itemize}

To prove : If for every sequence \(\left(x_n\right)\) in \(X\) that converges to \(x_0\), we have the sequence \(f\left(x_n\right)\) in \(Y\) converges to \(f\left(x_0\right)\) then \(f\) is continuous at \(x_0\) :
\begin{itemize}
    \item Conversely, suppose \(f\) is not continuous at \(x_0\).
    \item That is, there exists \(\epsilon>0\) for which \(f\left(\mathcal{N}_\delta\left(x_0\right)\right) \nsubseteq \mathcal{N}_\epsilon\left(f\left(x_0\right)\right)\) for all \(\delta>0\).
    \item In particular, \(f\left(\mathcal{N}_{\frac{1}{n}}\left(x_0\right)\right) \nsubseteq \mathcal{N}_\epsilon\left(f\left(x_0\right)\right)\) for all \(n \in \mathbb{N}\).
    \item So we can pick a sequence \(\left(x_n\right)\) in such a way that the following is satisfied: \(x_n \in \mathcal{N}_{\frac{1}{n}}\left(x_0\right)\) and \(f\left(x_n\right) \notin \mathcal{N}_e\left(f\left(x_0\right)\right)\) for all \(n \in \mathbb{N}\).
    \item Thus, \(x_n \rightarrow x_0\) but \(f\left(x_n\right) \nrightarrow f\left(x_0\right)\).
\end{itemize}

\begin{tcolorbox}
    Let's prove the contrapositve of \(B \implies A\)  
    \begin{equation*}
        \left( \exists (x_{n}) \subset X \right)(x_{n} \to x_{0} \vee f(x_{n} \not\longrightarrow f(x_{0}))  
    \end{equation*}    
    suppose \(f\) is not continous at \(x_{0} \in X\)  means that there exists \(\epsilon>0\) for which \(f\left(\mathcal{N}_\delta\left(x_0\right)\right) \nsubseteq \mathcal{N}_\epsilon\left(f\left(x_0\right)\right)\) for all \(\delta>0\). 

    then the following is also true 
    \begin{equation*}
        (\exists \epsilon > 0)(\forall n \in \mathbb{N})(f(\mathcal{N}_{\frac{1}{n}}(x_{0})) \not\subset \mathcal{N}_{\epsilon}(f(x_{0})))
    \end{equation*}    
    Pick \(x_{n} \in \mathcal{N}_{\frac{1}{n}}(x_{0})\) such that \(f(x_{n}) \notin \mathcal{N}_{\epsilon}(f(x_{0}))\)
    
    \(0 \leq d_{x}(x_{n},x_{0}) \leq \frac{1}{n}\)
    
    \((x_{n}) \to x_{0}\) and
    
    \( d_y(f(x_n),f(x0)) \geq \epsilon\)
    
    So \(f(x_{n})\) does not converge to \(f(x_{0})\)    
\end{tcolorbox}
\subsection{Continuity of a function over a set}
\begin{tcolorbox}
A function \(f: X \rightarrow Y\) is said to be continuous over \(D \subset X\)   if it is continuous at every \(x \in D\).
\end{tcolorbox}

\subsection{Continuity of a function}
\begin{itemize}
    \item A function \(f: X \rightarrow Y\) is said to be continuous if it is continuous at every \(x \in X\).
\end{itemize}
\subsection{Proposition 21}
\begin{itemize}
    \item A function \(f: X \rightarrow Y\) is continuous if, and only if, for each open set \(V\) of \(Y, f^{-1}(V)\) is open in \(X\) (where for any subset \(B\) of \(Y\), we define \(f^{-1}(B)=\{x: f(x) \in B\}\) ).
\end{itemize}

To prove: If a function \(f: X \rightarrow Y\) is continuous then for each open set \(V\) of \(Y, f^{-1}(V)\) is open in \(X\) :
\begin{itemize}
    \item - Suppose \(f: X \rightarrow Y\) is continuous.
    \item Consider any open set \(V\) of \(Y\). We claim that \(f^{-1}(V)\) is open in \(X\).
    \item To prove this, pick an arbitrary \(x \in f^{-1}(V)\).
    \item Since \(V\) is an open subset of \(Y\) and \(f(x) \in V\), there exists \(\epsilon>0\) such that \(\mathcal{N}_\epsilon(f(x)) \subseteq V\).
    \item By continuity of \(f\) at \(x\), there exists \(\delta>0\) such that \(f\left(\mathcal{N}_\delta(x)\right) \subseteq \mathcal{N}_\epsilon(f(x)) \subseteq V\).
    \item Thus we have \(\mathcal{N}_\delta(x) \subseteq f^{-1}(V)\).
\end{itemize}

To prove: If for each open set \(V\) of \(Y, f^{-1}(V)\) is open in \(X\) then \(f: X \rightarrow Y\) is continuous:
\begin{itemize}
    \item - Conversely, suppose \(f^{-1}(V)\) is open in \(X\) for every \(V\) open in \(Y\).
    \item We claim that \(f\) is continuous.
    \item To prove this, pick an arbitrary \(x \in X\) and an arbitrary \(\epsilon>0\).
    \item Since \(\mathcal{N}_\epsilon(f(x))\) is open in \(Y\), so \(f^{-1}\left(\mathcal{N}_e(f(x))\right)\) is open in \(X\) which along with the fact \(x \in f^{-1}\left(\mathcal{N}_\epsilon(f(x))\right)\) implies that there exists \(\delta>0\) such that \(\mathcal{N}_\delta(x) \subseteq f^{-1}\left(\mathcal{N}_\epsilon(f(x))\right)\) or equivalently, \(f\left(\mathcal{N}_\delta(x)\right) \subseteq \mathcal{N}_e(f(x))\).
\end{itemize}

Let's prove that if a function \(f: X \to Y\) is continuous implies 

\((\forall \text{ open set } V \subset Y)(f^{-1}(v)\) is open in \(X\)  )

\underline{\textbf{Proof}}

Consider any open set \(V \subset Y\)

Consider \(f^{-1}(v)\), if \(f^{-1}(v)\) is empty then it is open. If it is non-empty, then consider any \(x_{0} \in f^{-1}(v)\) 

Since we know that \(f\) is contionus, \(f\) is continuous at \(x_{0}\) and since \(f(x_{0}) \in V\) and \(V\) is open, so \(\exists \epsilon > 0 \) such that \(\mathcal{N} _{\epsilon}(f(x_{0}))\subset V\) and since \(f\) is continous at \(x_{0}\) \(\exists \delta > 0 \text{such that } f(\mathcal{N}_{\delta}(x_{0}))\subset \mathcal{N}_{\epsilon}(f(x_{0})) \subset V\)

So \(\mathcal{N}_{\delta}(x_{0}) \subset f^{-1}(V)\)  
\end{document}
