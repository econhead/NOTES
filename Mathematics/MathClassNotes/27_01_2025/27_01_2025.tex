\documentclass[12pt,a4paper]{article}

\usepackage[utf8]{inputenc}
\usepackage[T1]{fontenc}
\usepackage{parskip}
\usepackage{amsmath, amssymb, graphicx}
\usepackage{tcolorbox}
\usepackage{fancyhdr}
\setlength{\headheight}{15.6pt}
\pagestyle{fancyplain}
\fancyhead[L]{Laxman Singh}
\fancyhead[R]{\today}
\usepackage{float}
\floatstyle{boxed}
\restylefloat{figure}
\graphicspath{{/Users/econhead/NOTES/Mathematics/MathClassNotes/27_01_2025/Figures}}
\author{Laxman Singh}
\date{\today}
\title{Optimization}

\begin{document}
\section{Indirect Utility function}
\begin{tcolorbox}
    \begin{align*}
        \max_{(x,y) \in \mathbb{R}^2_{+}}& u(x,y)\\
        \text{s.t.}& \ p_{x}x+p_{y}y \leq M
    \end{align*}
\end{tcolorbox}
Solving the above probelm gives us the demand functions,
 \begin{align*}
    x^d(p_{x},p_{y},M)\\
    y^d(p_{x},p_{y},M)      
\end{align*}
The indirect utility function is defined as
 \begin{align*}
    V(p_{x},p_{y},M)=u(x^d(p_{x},p_{y},M),y^d(p_{x},p_{y},M))
\end{align*}
It gives us the optimal utility level at the price-income vector \((p_{x},p_{y},M)\)

\subsection{Properties of Indirect Utility function}
\begin{itemize}
    \item Indirect utility function is homogenous of degree \(0\) because demand is homogenous of degree\(0\).
    \item Indirect utility function is non-decreasing in income \(M\) 
    
    and non-increasing in prices \(p_{x},p_{y}\) because for \(m'>m''\),
    
    \( \mathcal{B}(p_x,p_y,M'') \subset \mathcal{B}(p_x,p_y,M') \)    
    
    and

    \(v(p_{x},p_{y},M') \geq v(p_{x},p_{y},M'')\)  
    \item Indirect utility function is quasi-convex.
    
    we want to prove that \(\{(p_{x},p_{y},M) \mid v(p_{x},p_{y},M) \leq  \bar{v} \}\) is a convex set for all \(\bar{v} \)
    
    \underline{\textbf{Proof}}

    Consider arbitrary \( \bar{v} \) and consider arbitrary \(\left( p_{x}',p_{y}',M' \right) \in A_{ \bar{v} } \) and \(\left( p_{x}'',p_{y}'',M'' \right) \in A_{ \bar{v} } \)  and arbitray \(\lambda \in [0,1]\) then we want to show 
    
    \(\lambda(p_{x}',p_{y}',M') + (1-\lambda)(p_{x}'', p_{y}'', M'') \in A_{ \bar{v}  }\)
    
    In other words we want to show that
    
    \(v(\lambda p_{x}' +(1-\lambda)p_{x}'', \ \lambda p_{y}' +(1-\lambda)p_{y}'', \ \lambda M' +(1-\lambda)M'')\leq  \bar{v}  \)  

    we know that \(v(p_{x}',p_{y}', M') \leq  \bar{v}  \) and   \(v(p_{x}'',p_{y}'', M'') \leq  \bar{v}  \)
    
    \(\mathcal{B}(\lambda p_{x}' +(1-\lambda)p_{x}'', \ \lambda p_{y}' +(1-\lambda)p_{y}'', \ \lambda M' +(1-\lambda)M'')\leq  \bar{v}  \) is our budget set.

    and we know that this inequality holds,

    \((\lambda p_{x}' +(1-\lambda)p_{x}'')x + (\lambda p_{y}' +(1-\lambda)p_{y}'')y \leq \lambda M' + (1-\lambda)M''\)  

    This tells us that any choice from out budget set \(\mathcal{B}\) that satisfy the above inequality also satisfies either 
    
    \(p_{x}'x+p_{y}'y \leq M'\)  or \(p_{x}''x+p_{y}''y \leq M''\) 
    
    this implies that 
    \begin{align*}
        v(\lambda p_{x}' +(1-\lambda)p_{x}'', \ \lambda p_{y}' +(1-\lambda)p_{y}'', \ \lambda M' +(1-\lambda)M'')\\
    \leq \max\left( v(p_{x}',p_{y}',M'),v(p_{x}'',p_{y}'',M'') \right)  \leq \bar{v} 
    \end{align*}  
\end{itemize}

\section{Expenditure Function}
\begin{tcolorbox}
     \begin{align*}
        \min_{(x,y)\in \mathbb{R}^2_{+}}& p_{x}x+p_{y}y\\
        \text{s.t.}& \ u(x,y) \geq  \bar{u}  
    \end{align*}
\end{tcolorbox}
Solving the above expenditure minimzation problem gives us the Hicksian demands,
 \begin{align*}
    x^h(p_{x},p_{y}, \bar{u} )\\
    y^h(p_{x},p_{y},  \bar{u} )  
\end{align*}

and the expenditure function is defined as follows 

\(e(p_{x},p_{y}, \bar{u} ) = p_{x}x^h(p_{x},p_{y}, \bar{u} ) + p_{y}y^h(p_{x},p_{y}, \bar{u} )   \)  

\subsection{Properties of the Expenditure function}
\begin{itemize}
    \item  The expenditure function is homogeneous of degree \(1\) in prices,
    
    \(e(\lambda p_{x},\lambda p_{y}, \mu ) = \lambda e(p_{x}, p_{y}, \mu)\)
    
    Note that the Hicksian demands are homogenous of degree \(0\) in prices because multiplying the objective in our expenditure minimzation problem by \(\lambda\), (where \(\lambda > 0\)) does not change the solution.
    
    \item The expenditure function is non-decreasing in \(\mu\) and it is also non-decreasing in prices \(p_{x},p_{y}\).
    
    we know that our expenditure minimization problem is
    \begin{align*}
        \min_{(x,y)\in \mathbb{R}^2_{+}}& p_{x}x+p_{y}y\\
        \text{s.t.}& \ u(x,y) \geq  \bar{\mu '}  
    \end{align*} 
    Now consider another satifaction level \(\mu '' \) such that \(\mu ' > \mu ''\) 

    \item The expenditure function is concave in prices.
\end{itemize}
\subsubsection{Kuhn-Tucker Optimization Problems}
\begin{tcolorbox}
    \begin{align*}
        \max_{(x,y)\in \mathbb{R}^{2}_{+}} & \quad \sqrt{x} + \sqrt{y} \\
        s.t. & \quad p_{x}x+p_{y}y \leq M\\
        & \quad x \geq 1\\
        & \quad y \geq 1
 \end{align*}
Assume that \(p_{x}+p_{y}<M \) 
\end{tcolorbox}
\begin{align*}
    &\mathcal{L}(x,y)=\sqrt{x}+\sqrt{y} - \lambda (p_{x}x+p_{y}y -M) + \mu_{x}(x-1) + \mu_{y}(y-1)\\
    &\frac{\partial \mathcal{L}}{\partial x}= \frac{1}{2\sqrt{x}} - \lambda p_{x} + \mu_{x} = 0\\
    &\frac{\partial \mathcal{L}}{\partial y}= \frac{1}{2\sqrt{y}} - \lambda p_{y} + \mu_{y} = 0\\
    &\lambda \geq 0, \quad p_{x}x+p_{y}y \leq M, \quad \lambda(p_{x}x+p_{y}y-M)=0\\
    &\mu_{x} \geq 0, \quad x \geq 1, \quad \mu_{x}(x-1)=0\\
    &\mu_{y} \geq 0, \quad y \geq 1, \quad \mu_{y}(y-1)=0\\
 \end{align*}
 Now if \(p_{x}x+p_{y}y <M\) then \(\lambda =0\) and \(\mu_{x}<0\) as well as \(\mu_{y}<0\) this rules out four of the eight possible cases.
 
 Now we only check for the cases \(p_{x}x+p_{y}y=M\)
 \begin{align*}
        \begin{array}{|c|c|c|c|}
           \hline x=1 & x=1 & x>1 & x>1\\
           \hline y=1 & y>1 & y=1 & y>1\\
           \hline \text{NP} & \mu_{y}=0 & & \\
           & y=\frac{M-p_{x}}{p_{y}}>1& &\\
           \hline
        \end{array}
\end{align*} 
\end{document}