\documentclass[12pt,a4paper]{article}

\usepackage[utf8]{inputenc}
\usepackage[T1]{fontenc}
\usepackage{parskip}
\usepackage{amsmath, amssymb, graphicx}
\usepackage{tcolorbox}
\usepackage{fancyhdr}
\setlength{\headheight}{15.6pt}
\pagestyle{fancyplain}
\fancyhead[L]{Laxman Singh}
\fancyhead[R]{\today}
\usepackage{float}
\floatstyle{boxed}
\restylefloat{figure}
\graphicspath{{/Users/econhead/NOTES/Mathematics/MathClassNotes/13_12_2024/Figures}}
\author{Laxman Singh}
\date{\today}
\title{Real Analysis}

\begin{document}
\section*{ \underline{Precise Definition of limit} } 
   Let \(f\) be a function defined on some open interval that contains \(a\), except possibly at a itself. Then we say that limit of \(f(x)\) as \(x\) approaches \(a \)  is \(L\), and we write 
    \begin{align*}
        \lim_{x \to a} f(x) = L 
   \end{align*}    
and the precise definiton would be, 
 \begin{align*}
    \left( \forall \epsilon > 0 \right) \left( \exists \delta > 0 \right)\left( \forall x \right) \left( 0 < |x-a|< \delta \implies |f(x)-L|< \epsilon \right)   
\end{align*}
The negation of the above definiton would be 
 \begin{align*}
    \left( \exists \epsilon > 0 \right) \left( \forall \delta > 0 \right)\left( \exists x \right) \left( 0< |x-a| < \delta \wedge |f(x)-L| \geq \epsilon \right)  
\end{align*}
The above statement tells us that if it is true then we have that,
 \begin{align*}
    \lim_{x \to a} f(x) \neq L
\end{align*}
and we say that limit of \(f(x)\) as \(x\) approaches \(a \)  is  not equal to \(L\).

\begin{figure}[ht]
    \centering
    \includegraphics[width=\textwidth]{limit.png}
    \caption{How to check for existence of limit graphically}
\end{figure}
 \subsubsection*{Example} 
 \begin{equation*}
     \lim_{x \to 0^+} \frac{1}{x} \quad \text{DNE}
 \end{equation*}    

\section{Subsets of Real line}
\(\mathbb{R}, \ \mathbb{N}, \ \mathbb{Z}, \ \mathbb{Q}, \ \mathbb{R}_{+}, \ \mathbb{Z}_{+}, \ \mathbb{Q}_{+}\)

 \begin{align*}
    \{x \in \mathbb{R} | a < x < b \}  = \left( a,b \right) \\
    \{x \in \mathbb{R} | a \leq x \leq b \}  = \left[ a,b \right] \\
    \{x \in \mathbb{R} | a \leq x < b \}  = [ a,b ) \\
    \{x \in \mathbb{R} | a < x \leq b \}  = ( a,b ] \\
\end{align*}

 \section{Bounds} 
 If \(X \subset \mathbb{R}\), then 

 \(c \in \mathbb{R}\) is an upper bound of \(X\) if \(c \geq x \quad \forall x \in X  \)
  
 examples of the sets which are bounded above are \( \ [0,1], \ \left( 0,1 \right) \)  but \(\mathbb{N}\) is not bounded above.  
 
 similiarly \(c \in \mathbb{R}\) is a lower bound  for \(X\) if \(c \leq x \quad \forall x \in X\)
 
 examples of some sets bounded below are \(\mathbb{N}, [0,1], (0,1)\).  

 We say that \(X \subset \mathbb{R}\) is bounded if it is bounded above and bounded below as well. So \(\mathbb{N}\) is not bounded but the sets \([0,1] \) and \((0,1)\) are bounded.
 
 If \(x \in X \) is an upperbound of \(X\) then \(x=\max X\).
 
 for example \(1= \max [0,1]\) but \(\max (0,1)\) does not exsist and similiarly \(\max \mathbb{N}\) does not exsist.    
 
  \subsection{Supremum} 
say \(X \subset \mathbb{R}\) and \(X\) is non empty  then \(\sup X\) is the lowest upperbound of \(X\) if the set is bounded above, and if the set is not bounded above then \(\sup X = \infty \)

for example, \(\sup (0,1) = 1\), \(\sup [0,1] = 1\), \(\sup \mathbb{N} = \infty \) and \( \sup \mathbb{Z} = \infty \).

But if the set \(X \) is empty then note that the every empty set is bounded above all \(x \in \mathbb{R}\) therefore \( \sup \phi = - \infty \).    

 \subsection{Infimum} 
for \(X \subset \mathbb{R}\) and \(X \neq \phi\), \(\inf X\)  is the greatest lower bound if \(X\) is bounded below, and it is \(- \infty\) if the set \(X\) is not bounded above.

Note that if \(X=\phi \) then \(\inf X = \infty\).

for example, \(\inf \mathbb{Z} = -\infty \), \( \inf \mathbb{N} = 1 \), \( \inf [0,1] = 0\) and \(\inf (0,1) = 0\).

 \section{Limits and Continuity of functions} 

 The question we want to answer is what happens to the value of the function as \(x\) approaches or gets closer to \(a\). 

 \begin{figure}[ht]
     \centering
     \includegraphics[scale=0.8]{4.png}
     \label{Label}
 \end{figure}
 Note \(f\) must be defined on some open interval around \(a\) (except possibly at \(a\) itself).  Here \(f:D \to \mathbb{R}\) where \(D \subset \mathbb{R}\).     
 
 Now the domain of \(f(x) \text{is} \ \mathbb{R}-1\)  
  \begin{align*}
     f(x) &= \frac{x^2 -1 }{x-1} \\
     \lim_{x \to 1} f(x) &= \lim_{x \to 1} \frac{(x-1)(x+1)}{x-1} = 2 
 \end{align*}

 So we say that \(l \in \mathbb{R}\) is the limit of \(f(x)\)  as \(x\) approaches \(a\)  and write \(l = \lim_{x \to a} f(x)\) if the following holds;
 \begin{itemize}
    \item \((\forall \epsilon > 0)(\exists \delta > 0)(\forall x \in D)(0 < |x-a| < \delta \implies |f(x) - l| < \epsilon)\).    
 \end{itemize}      
 
if we want to define \(\lim_{x \to a} f(x) \neq l\) we can negate the above definiton;
 \begin{align*}
    &\neg(\forall \epsilon > 0)(\exists \delta > 0)(\forall x \in D)(0 < |x-a| < \delta \implies |f(x) - l| < \epsilon) \\
    &(\exists \epsilon > 0)(\forall \delta > 0)( \exists x \in D)(0 <|x-a| < \delta \wedge |f(x) -l| \geq \epsilon)
\end{align*}   

In a smiliar way we can define the other related similiar concepts such as the left hand limit (LHL) and the right hand limit (RHL).

We say that \(\lim_{x \to a^{-}} f(x) =l \)  if 

\[(\forall \epsilon > 0)(\exists \delta > 0)(\forall x \in D)( a - \delta < x < a \implies |f(x) - l| < \epsilon )\] 

and We say that \(\lim_{x \to a^{+}} f(x) =l \)  if 

\[(\forall \epsilon > 0)(\exists \delta > 0)(\forall x \in D)( a < x < a + \delta \implies |f(x) - l| < \epsilon )\] 

again note that for the LHL to exist the fuction must be defined on some open interval to the left of \(a\) and similiary for the RHL to exist the function must be defined on some open interval to the right of \(a\) and not necessarily at \(a\).   

so for a limit to exist we need, 
\[ \lim_{x \to a} f(x) = l \ \text{if} \ \lim_{x \to a^{-}} f(x) = \lim_{x \to a^+} f(x) = l \]
more formally,
\begin{align*}
    &(\forall \epsilon > 0)(\exists \delta > 0)(\forall x \in D)(( a - \delta < x < a) \wedge 
    ( a < x < a +\delta) \implies |f(x) - l| < \epsilon ) \\
    &= (\forall \epsilon > 0)(\exists \delta > 0)(\forall x \in D)(0 < |x-a| < \delta 
    \implies |f(x) - l| < \epsilon)
\end{align*}

Now,

We say that \( \lim_{x \to a} \frac{1}{x} = \infty \) if 
\begin{equation*}
    (\forall \epsilon > 0)(\exists \delta > 0)(\forall x \in D)(0 < |x-a| < \delta \implies f(x) > \epsilon)
\end{equation*}      

We say that \(f\) is continous at \( x =a\) if \( \lim_{x \to a}f(x) = f(a)\).    

Note that at the endpoints of an interval for a function to be continous we only need \( \lim_{x \to a^+}f(x) = f(a)\) or \( \lim_{x \to a^{-}}f(x) = f(a)\) given the left or right endpoint respectively.

\section{Sequences of Real Numbers} 
We are always talking about infinte sequences when we are dealing with sequnces of real numbers becaue it contains countably infinite terms.

\(x_{1},x_{2},x_{3},x_{4},\ldots \)

Formally a sequence \(x_{n} \) or \(x(n)\) is a function defined as \( x:\mathbb{N}\to\mathbb{R}\), for example,

\(x_{n}=\frac{1}{n}\), \(x_{n} = (-1)^n\), etc,. 

\subsection{Limit of a sequence}
We say that a sequence of reals \((x_{n})\) is convergent if there exists a number \(l \in \mathbb{R} \) such that ,
\begin{equation*}
    (\forall \epsilon > 0 )(\exists N \in \mathbb{N})(\forall n \in \mathbb{N})(n > N \implies |x_{n} - l| < \epsilon)
\end{equation*}    
and this number \(l\) is known as the limit of the sequence \((x_{n})\), which is written as \(\lim_{n \to \infty} x_{n} =l \) or \(x_{n} \to l\).
\begin{figure}[ht]
    \centering
    \includegraphics[scale=0.5]{5.png}
    \caption{A few examples}
\end{figure}

Now to show that \(1 \neq \lim_{n \to \infty }(-1)^n\) we can show,
\begin{equation*}
    (\exists \epsilon > 0)(\forall N \in \mathbb{N})(\exists n \in \mathbb{N})(n > N \wedge |x_{n} - l| \geq \epsilon)
\end{equation*}      

 \subsubsection*{Propositon 1} 
 \begin{itemize}
    \item A sequence cannot have more than one limit.       
 \end{itemize}
    \underline{Proof:} Suppose a sequence has two differnet limits \(a\) and \(b\) and it is apporaching both \(a\) and \(b\), 

    Now by definiton of the limit of a sequence,
    \begin{equation*}
        (\forall \epsilon > 0)(\exists N \in \mathbb{N})( \forall n \in \mathbb{N})(n > N \implies |x_{n} - l| < \epsilon)
    \end{equation*} 

 \begin{figure}[h]
    \centering
    \includegraphics[scale=0.7]{6.png}
    \caption{Graphical Proof}
\end{figure}

 \subsubsection*{Proposition 3} 
 \begin{itemize}
    \item Let \(\left(x_n\right)\) and \(\left(y_n\right)\) be convergent sequences in \(\mathbb{R}\) and \(x_n \leq y_n\) for infinitely many \(n\). Then \(\lim x_n \leq \lim y_n\)  
 \end{itemize}

  \subsubsection*{Propositon 4 (Squeeze Theorem} 
  \begin{itemize}
    \item Let \(\left(x_n\right)\) and \(\left(y_n\right)\) and \(\left(z_n\right)\) be sequences in \(\mathbb{R}\) and \(x_n \leq y_n \leq z_n\) for almost all \(n\). If \(\lim x_n=\lim z_n=a\), then \(\left(y_n\right)\) converges to \(a\).
  \end{itemize}

 \subsection{Bounded Sequences} 
 Bounded sequence of real numbers:
\begin{itemize}
    \item We say that a real sequence \(\left(x_n\right)\) is bounded from above if there exists a real number \(K\) with \(x_n \leq K\) for all \(n=1,2, \ldots \).
    \item This is equivalent to saying that
    \begin{align*}
        \sup \left\{x_n \mid n \in \mathbb{N}\right\}<\infty
    \end{align*}
    \item Dually, \(\left(x_n\right)\) is said to be bounded from below if \(\inf \left\{ x_n: n \in \mathbb{N}\right\}>-\infty \)
\end{itemize}
\begin{tcolorbox}
    \(\left(x_n\right)\) is called bounded if it is bounded from both above and below, that is,
\begin{align*}
\sup \left\{\left|x_n\right| \mid n \in \mathbb{N}\right\}<\infty
\end{align*}
\end{tcolorbox}
\begin{figure}[ht]
    \centering
    \includegraphics[scale=0.7]{7.png}
    \caption{Proof of Boundedness}
\end{figure}
\end{document}