\documentclass[12pt,a4paper,fleqn]{article}

\usepackage[utf8]{inputenc}
\usepackage[T1]{fontenc}
\usepackage{parskip}
\usepackage{amsmath, amssymb, graphicx}
\usepackage{tcolorbox}
\usepackage{fancyhdr}
\setlength{\headheight}{15.6pt}
\pagestyle{fancyplain}
\fancyhead[L]{Laxman Singh}
\fancyhead[R]{\today}
\usepackage{float}
\floatstyle{boxed}
\restylefloat{figure}
\author{Laxman Singh}
\date{\today}
\title{Hypothesis Testing}
\graphicspath{{/Users/econhead/NOTES/Probability/ProbClassNotes/27_10_2024/Figures}}
\begin{document}
\section{\underline{Hypothesis Testing}}
  \subsection{Example 1}  
We have two hypotheses, a default (null) (\(H_{0}\)) hypothesis and some challenging (alternative) (\(H_{1}\) or \(H_{A}\)) hypothesis.
Say we have an experiment of tossing a coin 100 times and two hypotheses that the coin is fair \(H_{0}\)  and the coin is not fair \(H_{A}\).  
\begin{align*}
    H_{0} & : \theta =\frac{1}{2}\\
    H_{A} & : \theta \neq \frac{1}{2}\\
    \text{Test} & : \text{Possible Observations} \to \{H_{0},H_{A}\}
\end{align*}
A test will classify the observations of an experiment according to our two possible hypotheses.
As statisticians our job is to design a sensible test.
A sensible test in our example could be \[|X - 50| \leq t \to H_{0}\]
where \(X\) is the number of Heads.  

Now under the null Hypothesis, \(X \sim \text{Bin}(100,\frac{1}{2}) \), and \(X \overset{\cdot}{\sim} \mathcal{N}(50,25)\) and \(\frac{X-50}{5} \overset{\cdot}{\sim} \mathcal{N}(0,1)\) then under this a sensible type 1 error probability would be,
\begin{align*}
    \Pr_{H_{0}}(|X-50| > t) &=0.05\\
    \Pr (|Z|>2) &= 0.05 \\
    \Pr \bigg( \left|\frac{X-50}{5}\right| > 2 &= 0.05\bigg) \\
    \Pr (|X-50| > 10) &= 0.05
\end{align*}  
So in this case we accept \(H_{0}\) for \(40,41,\ldots,50,\ldots,60\) and for the other values we reject the null \(H_{0}\).  

 \subsection{Example 2} 

Suppose you draw a sample of size \(1\) and we have the following two hypotheses,
\begin{align*}
    H_{0} &: X \sim \text{Unif}(0,1)\\
    H_{1} &: X \sim \text{Beta}(2,1)\\
\end{align*} 
We need to find a sensible test for the \(\alpha = 0.1\) level of significance, and also find the type 2 error probability; 

\(T : [0,1] \to \{H_{0},H_{1}\} \)  will be the sensible test we are looking for;
 \begin{align*}
    X\leq t \qquad \implies \ \text{Accept} \ H_{0} \\
    X> t \qquad \implies \ \text{Accept} \ H_{1}
\end{align*}
\begin{figure}[ht]
    \centering
    \includegraphics[scale=0.8]{1.png}
\end{figure}

Since \(\alpha=0.1\) we know that \(\Pr\left( \text{Type 1 Error} \right) \ = 0.1 \) now we can solve for \(t\);
 \begin{align*}
    &\int_{t}^{1} 1 \mathrm{d}x = 0.1 \\
    \implies &t = 0.9
\end{align*}
And now we can solve for \(\Pr\left( \text{Type 2 Error} \right) \);
 \begin{align*}
    \Pr\left( \text{Type 2 Error} \right) \ &= \Pr_{H_{1}}(X \leq 0.9) \\
    &= \int_{0}^{0.9} 2x \mathrm{d}x = 0.81 
\end{align*}  
\pagebreak
 \subsection{Example 3} 
Suppose \(X\) is one observation,
\begin{align*}
    H_{0} &: X \sim \text{Unif}(0,1)\\
    H_{1} &: f_{X}(x)=\begin{cases} 4x \ \text{for} \ x \in [0,\frac{1}{2}]\\
        4-4x  \ \text{for} \ x \in (\frac{1}{2},1] 
    \end{cases}\\
    \alpha = 0.1    
\end{align*} 
Find a sensible test for the \(\alpha \) level of significance.

\begin{figure}[ht]
    \centering
    \includegraphics[scale=0.8]{2.png}
\end{figure}

\( \implies \) Reject if \(0.45<X<0.55\), Accept \(H_{0}\) otherwise.  

And the probability of type 2 error would be,
 \begin{align*}
    &\int_{0}^{0.45} 4x \mathrm{d}x + \int_{0.55}^{1} 4-4x \mathrm{d}x \\
    &= 2 \int_{0}^{0.45} 4x \mathrm{d}x = {\left[ 4x^2 \right]}_{0}^{0.45}\\
    &=4{(0.45)}^2 = {(0.9)}^2 = 0.81
\end{align*}
\pagebreak
 \subsection{Example 4} 
Suppose \(X_{1},X_{2},\ldots,X_{n} \ iid \ \mathcal{N}(\theta,1), \quad n = 25 \).  
\begin{align*}
    H_{0} &: \theta =0\\
    H_{1} &: \theta =1\\
    \alpha &= 0.05 
\end{align*}
Note that, 
\begin{align*}
    \overline{X}_{25} \sim \mathcal{N}\left( 0,\frac{1}{25} \right) \quad \text{Under} \ H_{0}\\
    \overline{X}_{25} \sim \mathcal{N}\left( 1,\frac{1}{25} \right)
    \quad \text{Under} \ H_{1}\\ 
\end{align*}
A sensible test in this example would be \(T : (-\infty, \infty) \to \{H_{0},H_{1}\} \)  
 \begin{align*}
    &\text{Accept} \ H_{0} \ \text{if} \ \ \overline{X}_{25} \leq 0.329 \\
    &\text{Reject otherwise} 
\end{align*}
Because,
 \begin{align*}
    \Pr_{H_{0}}\left( \overline{X}_{25} >t  \right) = 0.05 \\
    \Pr_{H_{0}}\left(  5\overline{X}_{25} >5t  \right) = 0.05 \\
    5t=1.645\\
    t= \frac{1.645}{5}= 0.329 
\end{align*}
\end{document} 